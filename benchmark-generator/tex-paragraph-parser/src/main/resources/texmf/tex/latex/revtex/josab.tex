%%% ======================================================================
%%%  @LaTeX-file{
%%%     filename        = "josab.tex",
%%%     version         = "3.0",
%%%     date            = "October 20, 1992",
%%%     ISO-date        = "1992.10.20",
%%%     time            = "15:41:54.18 EST",
%%%     author          = "Optical Society of America",
%%%     contact         = "Frank E. Harris",
%%%     address         = "Optical Society of America
%%%                        2010 Massachusetts Ave., N.W.
%%%                        Washington, D.C.  20036-1023",
%%%     email           = "fharris@pinet.aip.org (Internet)",
%%%     telephone       = "(202) 416-1903",
%%%     FAX             = "(202) 416-6120",
%%%     supported       = "yes",
%%%     archived        = "pinet.aip.org/pub/revtex,
%%%                        Niord.SHSU.edu:[FILESERV.REVTEX]",
%%%     keywords        = "REVTeX, version 3.0, sample, Optical
%%%                        Society of America",
%%%     codetable       = "ISO/ASCII",
%%%     checksum        = "61879 650 4558 30808",
%%%     docstring       = "This is a sample JOSA B paper under REVTeX
%%%                        3.0 (release of November 10, 1992).
%%%
%%%                        The checksum field above contains a CRC-16
%%%                        checksum as the first value, followed by the
%%%                        equivalent of the standard UNIX wc (word
%%%                        count) utility output of lines, words, and
%%%                        characters.  This is produced by Robert
%%%                        Solovay's checksum utility."
%%% }
%%% ======================================================================
%%%%%%%%%%%%%%%%%%% file josab.tex %%%%%%%%%%%%%%%%%%%%
%                                                     %
%   Copyright (c) Optical Society of America, 1992.   %
%                                                     %
%%%%%%%%%%%%%%%%%% October 20, 1992 %%%%%%%%%%%%%%%%%%%
%
\documentstyle[osa,manuscript]{revtex}  % DON'T CHANGE
%
%
\newcommand{\MF}{{\large{\manual META}\-{\manual FONT}}}
\newcommand{\manual}{rm}        % Substitute rm (Roman) font.
\newcommand\bs{\char '134 }     % add backslash char to \tt font
%
%
\begin{document}                % INITIALIZE - DONT CHANGE
%
%
%
\title{Generation, propagation, and amplification of dark solitons}
%
\author{W. Zhao and E. Bourkoff}
%
\address{Department of Electrical and Computer Engineering,
The University of South Carolina,
Columbia, South Carolina, 29208}
%
\maketitle
\begin{abstract}
The technique for generating dark solitons with constant background using
guided-wave Mach--Zehnder interferometers is further examined. Under
optimal conditions, a reduction of 30\% in both the input optical power and
the driving voltage can be achieved, as compared with the case of complete
modulation.  Dark solitons are also found to experience compression through
amplification.  When the gain coefficient is small, adiabatic amplification
is possible.  Raman amplification can be used as the gain mechanism for
adiabatic amplification, in addition to being used for loss-compensation.
The frequency and time shifts caused by intrapulse stimulated Raman
scattering are both found to be a factor of 2 smaller than those for bright
solitons.  Finally, the propagation properties of even dark pulses are
described quantitatively.
\end{abstract}


\section{ INTRODUCTION}
\label{INT}
Nonlinear optical pulses can propagate in dispersive fibers in the form of
bright and dark solitons under certain conditions, as first described by
Zakharov and Shabat in 1972\cite{ZA} and in 1973,\cite{ZB} respectively.
They are stationary solutions of the initial boundary value problem of the
nonlinear Schr{$\rm\ddot o$}dinger equation (NLSE).\cite{SA} In the
anomalous dispersion regime of the fiber, under the boundary condition $
u( z, t = \pm \infty ) = 0 $, there exists a class of particle-like,
stationary solutions called bright solitons.\cite{HA} In the normal
dispersion region, under the
\begin{center}
{\small  \copyright\ Optical Society of America, 1992.}
\end{center}
boundary condition $ | u( z, t = \pm \infty ) | = $constant, one can obtain
another class of stationary solutions, which are called dark solitons,
since a dip occurs at the center of the pulse.\cite{HB} Ever since the
pioneering work by Zakharov and Shabat\cite{ZA,ZB} and Hasegawa and
Tappert,\cite{HA,HB} optical solitons have been an active topic of
research. This is particularly true since advances in experimental
techniques for generating ultrashort pulses in the picosecond regime have
made it possible to observe soliton effects in single-mode optical fibers.
The bright soliton was first successfully observed in single-mode optical
fibers by Mollenauer {\it et al.\ } in 1980,\cite{MA} and the dark soliton
was  first observed by Emplit {\it et al.\ } in 1987.\cite{EA}


The characteristics of bright solitons have been studied extensively during
the past decade.\cite{AA,MB} It was found\cite{SA} that bright solitons are
periodic and highly stable against small perturbations, such as fiber loss,
background noise, and amplitude variations.\cite{HA,HB} Ideally, when fiber
loss is neglected, the fundamental bright soliton can propagate inside an
anomalously dispersive fiber over an infinitely long distance without
changing its pulse shape.  This can occur because, for a fundamental
soliton, the effect of dispersion on the pulse is exactly balanced by that
of the nonlinear refractive index of the fiber, i.e., the self-phase
modulation.  Solitons can also survive collisions between them.  The
interaction force between two neighboring solitons is periodic and
decreases exponentially with their separation.\cite{GA} Another
characteristic of bright solitons is that they can be adiabatically
amplified under certain conditions when gain is introduced into the fiber,
e.g., through Raman amplification.\cite{BA} The effect of fiber loss on the
pulse can thus be compensated for by injecting a cw laser beam at a shorter
wavelength into the fiber, whereby stimulated Raman scattering transfers
its energy to the soliton.\cite{HC} Therefore solitons are  candidates for
information carriers for future optical communications.  Much research has
been done in this area.\cite{DA} The possibility of stable, repeaterless,
all-optical soliton transmission at a 10--GHz rate across almost 5000 km
has been numerically demonstrated\cite{HD,MC} and experimentally realized
with a rate-length product of approximately 11,000 GHz km.\cite{MD} More
recently, with erbium-doped fiber amplifiers, soliton transmission of 9,000
km at 4 Gbits/s has been realized.\cite{ME}


Because a dark pulse (with a dip of pulse intensity under constant
background),\cite{EA,KA,WA} especially the so called odd dark pulse (for
which the electric field changes sign at the center of the pulse), cannot
be easily generated, dark solitons have been studied less than their
counterparts, bright solitons.  However, as a result of recent developments
in techniques for synthesizing short optical pulses with almost arbitrary
shapes and phases,\cite{WB} it is possible to observe soliton like
propagation of individual dark pulses in single-mode fibers. Because these
fibers exhibit normal dispersion over a large spectral region, extending
from UV to  IR ($ \lambda < 1.3 \mu\rm m $), many cw and pulsed laser
sources can be used to generate dark solitons.  As a result,  dark solitons
have attracted increasing attention. \ldots


In the following discussions, we adopt the normalization convention used in
Agrawal's book.\cite{AB} We normalize the field amplitude $A$ (optical
power $P_0 = A^2 $) into $u$ by
\begin{eqnarray*}
u = \left( { 2 \pi n_2 {\tau_0}^2 }\over
{ \lambda A_{\rm eff} | \beta_2 | } \right)^{1/2} A ,
\end{eqnarray*}
where $A_{\rm eff }$ is the effective area of the propagating mode, $n_2 =
3.2\times 10^{-16}$cm$^2 /$W is the nonlinear optical Kerr coefficient of
the silica fiber, and $ \beta_{2} $ is a parameter describing the group
velocity dispersion of fiber, defined as the second-order derivative of the
propagation constant with respect to the radiant frequency evaluated at the
signal frequency. The time variable $ t $ is normalized by a characteristic
time constant $ \tau_0 $ (e.g., $ \tau_{\rm FWHM}  =  1.76 \tau _{0} $ for
hyperbolic secant pulses), and the spatial variable $z$ is normalized by
the so-called dispersion length,
\begin{eqnarray*}
L_D  &=&  {{\tau_0}^2}\over{\beta_2 } .
\end{eqnarray*}
As an example, at wavelength $ \lambda  =  1.06 \, \mu$m with $ A_{\rm eff}
=  40\, \mu {\rm m}^2$, for a pulse with $ \tau _{0}  =  1\, $ps, the
normalized distance $z  =  1 $ corresponds to a real fiber length of $
L_{D}= 60\,$m, and $ u  =  1 $ represents an optical power of $
P_{0}=3.5\,$W.  However, when $ \tau _{0} =0.1\,$ps, $ L_{D }=60\,$cm, and
$ P_{0}=350\,$W.



\section{GENERATION OF DARK SOLITONS}
\label{GDS}
In our earlier work\cite{ZBD,ZBE} we discussed the possibility of using an
integrated Mach--Zehnder interferometer (MZI) to generate dark solitons
with constant background.  The idea is to drive a broad bandwidth MZI with
a square-shape electric voltage with picosecond rise time.  The applied
electric voltage signal introduces a relative phase shift, proportional to
the voltage, between the two arms of the interferometer by means of the
electro-optic effect of the waveguide material.   At the output,  the two
components of light are recombined, and the resultant optical field is
proportional to the cosine of half of the total phase difference, the
induced relative phase shift plus any other static (residual) phase
differences. Therefore the pulse after the MZI, when properly biased, can
have the form
\begin{eqnarray}
u (0,t)  =  a\, {\rm sin} [ \delta \pi /2\, {\rm tanh} (t) ],
\label{E1}
\end{eqnarray}
where $a$ is the field amplitude of the input cw laser beam and  $ \delta $
the ratio of the applied voltage, approximated by a hyperbolic tangent
function of time,  to the half-wave voltage of the MZI.
\ldots

\ldots
The bandwidth requirement of the MZI is determined by the desired pulse
duration, which is approximately half of the reciprocal of pulse duration.
For a 50 ps dark soliton, 10 GHz is required, and this is achievable by
current technology.



\section{PROPAGATION AND AMPLIFICATION}
\label{PAA}
As discussed in Section \ref{GDS}, when smaller values of $ \delta $ are
used, pulses of better characteristics are obtained.  This can be seen in
Fig. 1(d),  where $ a  =  1.33 $ and a pure fundamental dark soliton is
generated. \ldots .   \ldots In what follows, we will examine the
possibility of amplification and compression of dark solitons with a
constant gain and show that the stimulated Raman scattering can be used to
amplify dark solitons as well as to compensate for the fiber loss.

We first examine the solution of a modified NLSE with a constant gain:
\begin{eqnarray}
i u_{z} - {1/2} u_{tt} + |u|^2 u  =  i \Gamma u,
\label{E2}
\end{eqnarray}
where $\Gamma $ is assumed to be a constant, appropriate for the Raman
amplification under strong pumping without depletion.  The solution of a
similar equation to Eq. (\ref{E2}), but in the anomalous dispersion regime,
in which bright solitons are amplified and compressed by the gain, has been
analyzed by Blow {\it et al}.\cite{BA} To solve the equation,  we make the
following variable transformation:
\begin{mathletters}
\begin{eqnarray}
t'  &=&  t e^{ \Gamma z }, \label{E4}  \\
z'  &=&  { e^{2 \Gamma z } - 1 \over  2 \Gamma },  \label{E5}  \\
u   &=&  v e^{ \Gamma z } .                          \label{E6}
\end{eqnarray}
\end{mathletters}
Under this transformation,  the NLSE has the new form
\begin{eqnarray}
i v_{z'} -\slantfrac{1}{2} v_{ t' t' } - |v|^2 v  &=&
- { \Gamma t' \over 2 \Gamma z' + 1} v_{t'}. \label{E7}
\end{eqnarray}
The solution of Eq. (\ref{E2}) when $\Gamma $ = 0 is well known and has the
form ${\rm exp} [i \sigma (z,t) ] \kappa \tanh \kappa t $, where $\kappa $
is the form factor and the phase variable satisfies $ \partial \sigma /
\partial z  =  \kappa^2 $.\cite{ZA} Therefore, when the right-hand-side of
Eq.(\ref{E7}) is zero, an exact solution for $v(z',t)$ can be obtained from
Eq. (\ref{E7}). On the other hand, when $z \rightarrow \infty $ and hence
$z' \rightarrow \infty $ or $ \Gamma \rightarrow 0$,  the right-hand side
of Eq. (\ref{E7}) becomes infinitely small. Under these conditions, we can
treat the right-hand side of Eq. (\ref{E7}) as a perturbation to the NLSE.
\ldots
\begin{eqnarray}
u(z,t)&=&{\rm exp}\left( i{e^{2\Gamma z}-1 \over 2\Gamma}\right)
e^{\Gamma z} \, {\rm tanh} (te^{\Gamma z}),
\label{E8}                                \\
\Gamma&=&g(e^{-2\Gamma_pz} + e^{-2\Gamma_p(L-z)}) - \Gamma_s,
\label{E9} \\
g&=&{\Gamma_p(\Gamma_s + \beta)L \over {\rm sinh}(\Gamma_pL)}
e^{\Gamma_pL} , \label{E10} \\
\kappa(z) &=& \kappa_0 \, {\rm exp}(\beta z). \label{E11}
\end{eqnarray}
\ldots

In summary, we have studied the propagation properties of dark solitons
under the influence of gain.  The dark soliton can be amplified and
compressed adiabatically when the gain coefficient remains small, e.g., $
\Gamma < 0.1 $.  As the gain increases above this value, secondary gray
solitons will be generated.  Stimulated Raman scattering can be utilized to
provide the gain. When the product $ \Gamma _p L $ is kept small, dark
solitons can be amplified adiabatically with high quality. Such a property
of dark solitons enable us to obtain dark solitons with short durations for
the ease of observation and transmission.


\section{EFFECTS OF INTRAPULSE STIMULATED RAMAN SCATTERING}
\label{EIS}
The properties of dark solitons considered thus far are based on the
simplified propagation equation (\ref{E2}). When the pulse duration reaches
the subpicosecond regime, it becomes necessary to include higher-order
nonlinear and dispersive effects.\cite{ZBF} These effects represent
higher-order terms in the derivation of wave equation from the Maxwell's
equations.  Intrapulse stimulated Raman scattering (ISRS) is one of the
dominating effects.  It causes soliton self-frequency shift for both bright
solitons\cite{MG,GD} as well as for dark solitons.\cite{WD} Since its
discovery for bright solitons,\cite{ZBF} considerable attention has been
paid to such effects.
\ldots

The effect of ISRS on bright solitons  is to shift in both the temporal and
spectral domains.  It has been demonstrated that the frequency red shift of
bright solitons is linear with propagation length, at a rate of $-8t_d /15
{\tau_0}^4 $ per unit propagation distance, where $ t_d $ is the delay time
the nonlinear response of the medium (typically 6 fs) and $ \tau _{0} $ is
the normalized soliton duration.  The temporal shift is a direct
consequence of the group velocity dispersion of the fiber. The temporal
shift was found to be $4 t_d z^2 /15 {\tau_0}^4 $.\cite{BB} Note that the
shifting rate is proportional to $|u|^4 $ because $ \tau _0 $ is the
inverse of the normalized amplitude. ISRS is especially pronounced for high
peak power pulses. Therefore, when a higher-order soliton is launched, the
ISRS will cause soliton fission.\cite{TA} Because of such effects, the
initially bound state ceases to exist and solitons of different amplitudes
are separated from one another.  The energies of these separating solitons
are distributed in such way to ensure conservation of momentum.
\ldots


\begin{eqnarray}
iu_z - {1/2}u_{tt}+|u|^2u &=& \tau_d{\partial |u|^2 \over \partial t}u,
\label{E12}
\end{eqnarray}
\ldots

We introduce a simple model of the shift for fundamental dark solitons:
\begin{mathletters}
\begin{eqnarray}
{{\rm d}\omega \over {\rm d}z}  &=& {4\tau_d \over 15}\kappa^4,
\label{E13}\\
{{\rm d}\theta \over {\rm d}z}  &=& \omega.
\label{E14}
\end{eqnarray}
\end{mathletters}
\ldots



We next study the behavior of dark solitons when both adiabatic
amplification and ISRS are present. Figure 6(a) shows the pulse shape of a
fundamental dark soliton in such a case.  In this case the fundamental dark
soliton loses its amplitude contrast, as it does in Fig. 4(a), and the ISRS
temporal shift is enhanced by the effect of adiabatic gain. In the simple
model described by Eqs. (\ref{E8}) and (\ref{E13}), the temporal shift by
ISRS has the functional form
\begin{eqnarray}
\theta  =  {\tau_d \over 60 \Gamma^2 } ( e^{ 4 \Gamma z } - 1 - 4
\Gamma z ) .             \label{E15}
\end{eqnarray}
\ldots

In summary, the ISRS causes a shift of dark solitons.  A salient feature of
dark solitons is that the rate of such shift is half the value for bright
solitons, when the slow loss of contrast is neglected.  This leads to a
better stability of a fundamental dark soliton against such perturbations.
However, the situation for higher-order dark solitons is more complicated
because there are amplitude changes associated with each soliton.  The
symmetry of higher-order solitons is broken.  Red secondary solitons gain
energy at the expense of blue ones. The primary soliton ceases to be a
fundamental dark soliton and suffers energy losses and frequency blue
shift.


\section{EVEN  DARK PULSES}
\label{EDP}
Even dark pulses,\cite{KA,WA} which are symmetric functions of time
centered around the pulse, can be simply generated by driving the MZI with
a short electric pulse.  In this case, only an intensity modulation that
gives a dip of optical power under the constant background is required.

The propagation characteristics of even dark pulses are different from
those of odd dark solitons. Generally, even dark pulses  split into pairs
of secondary dark solitons without the formation of a primary dark soliton.
The energy of the input dark pulse is then redistributed into a certain
number of paired secondary dark solitons.  Secondary dark solitons, which
are called gray solitons\cite{TB} have nonzero intensity of pulse
centers.
\ldots


If we define the amplitudes of the secondary soliton pairs as
\begin{eqnarray}
\kappa_n  =  \kappa_0 - \Delta_{n} ,  \label{E16}
\end{eqnarray}
then the $n$th order secondary pulse shape (n = 1, 2, 3, \ldots )
has the form
\begin{eqnarray}
u_n (z,t)  =  \kappa_{0}{(\lambda_n - i \nu_n )^2 - \nu_n
\,{\rm exp} [ 2 \nu_n (t-t_{n0} - \lambda_{n} z)] \over  1 +
\nu_n\, {\rm exp} [ 2 \nu_n (t-t_{n0} - \lambda_n z)]} e^{iz},
\label{E17}
\end{eqnarray}
\ldots


\section{CONCLUSIONS}
We have discussed the possibility of using the waveguide Mach--Zehnder
interferometer to generate a variety of dark solitons under constant
background.  Under optimal operation,  30\% less input power and driving
voltage are required than for complete modulation.  The generated solitons
can have good pulse quality and stimulated Raman scattering process can be
utilized to compensate for fiber loss and even to amplify and compress the
dark solitons.  Generally speaking, when a constant gain coefficient is
included in the NLSE, adiabatic amplification of the dark soliton is
possible, as long as the gain $\Gamma $ is kept small
\ldots

When a fundamental dark soliton is adiabatically amplified in the presence
of ISRS, the spectral shift and thus the temporal shift follow a simple
rule, Eq. (\ref{E15}), which takes into consideration the exponentially
increasing amplitude and linear dependence of the shift on the propagation
distance.  We find that such a simple model can accurately describe the
behavior of fundamental dark solitons subject to adiabatic amplification
and ISRS.  The propagation properties of even dark pulses are also studied,
with special attention to the distribution of energies among secondary gray
solitons.  Despite their more complicated nature, our results demonstrate
that the partition of the energy is similar for quite different input pulse
shapes, as long as they have the same background intensity and total energy
for the input pulse.  One can use the partition rule obtained here to
predict the amplitude of secondary solitons produced from an input even
dark pulse.



\acknowledgments

The authors  thank the reviewers for their constructive comments.
This research was supported by National Science Foundation grant
ECS-91960-64.


\begin{references}
\bibitem{ZA}
V. E. Zakharov and A. B. Shabat, ``Exact theory of two-dimensional
self-focusing and one-dimensional self-modulation of waves in nonlinear
media,'' Sov. Phys. JETP {\bf 5,} 364--372 (1972).
\bibitem{ZB}
V. E. Zakharov and A. B. Shabat, ``Interaction between solitons in a stable
medium,'' Sov. Phys. JETP {\bf 37,} 823--828 (1973).
\bibitem{SA}
J. Satruma and N. Yajima, ``Initial value problems of one-dimensional
self-phase modulation of nonlinear waves in dispersive media,'' Progr.
Theor. Phys. Suppl. {\bf 55,} 284--305 (1974).
\bibitem{HA}
A. Hasegawa and F. Tappert, ``Transmission of stationary nonlinear optical
pulses in dispersive dielectric fibers. I. Anomalous dispersion,'' Appl.
Phys. Lett. {\bf 23,} 142 (1973).
\bibitem{HB}
A. Hasegawa and F. Tappert, ``Transmission of stationary nonlinear optical
pulses in dispersive dielectric fibers. II. Normal dispersion,'' Appl.
Phys. Lett. {\bf 23,} 172 (1973).
\bibitem{MA}
L. F. Mollenauer, R. H. Stolen, and J. P. Gordon, ``Experimental
observation of pico-second pulse narrowing and solitons in optical
fibers,'' Phys. Rev. Lett. {\bf 45,} 1095 (1980).
\bibitem{EA}
P. Emplit, J. P. Hamaide, R. Reynaud, C. Froehly, and A. Barthelemy,
``Picosecond steps and dark pulses through nonlinear single mode fibers,''
Opt. Commun. {\bf 62,} 374--379 (1987).
\bibitem{AA}
S. A. Akhmanov, V. A. Vysloukh, and A. S. Chirkin, ``Self-action of wave
packets in a nonlinear medium and femtoseconds laser pulse,'' Sov. Phys.
Usp. {\bf 29,} 642--677 (1986).
\bibitem{MB}
L. F. Mollenauer, ``Solitons in optical fibers and the soliton laser,''
Philos. Trans. Roy. Soc. Lond. A {\bf 315,} 437--450 (1985).
\bibitem{GA}
J. P. Gordon, ``Interaction forces among solitons in optical fibers,'' Opt.
Lett. {\bf 8,} 596--598 (1983).
\bibitem{BA}
K. J. Blow, N. J. Doran, and David Wood, ``Generation and stabilization of
short soliton pulses in amplified nonlinear Schr$ roman o dotdot $dinger
equation,'' J. Opt. Soc. Am. B {\bf 5,} 381--91 (1988).
\bibitem{HC}
A. Hasegawa, ``Amplification and reshaping of optical solitons in a glass
fiber -- IV: Use of the stimulated Raman process,'' Opt. Lett. {\bf 8,}
650--652 (1983).
\bibitem{DA}
N. J. Doran and K. J. Blow, ``Solitons in optical communications,'' IEEE J.
Quantum Electron. {\bf QE-19,} 1883--1888 (1983).
\bibitem{HD}
A. Hasegawa, ``Numerical study of optical soliton transmission amplified
periodically by the stimulated Raman process,'' Appl. Opt. {\bf 23,}
3302--3309 (1984).
\bibitem{MC}
L. F. Mollenauer, J. P. Gordon, and M. N. Islam, ``Soliton propagation in
long fibers with periodically compensated loss,'' IEEE J. Quantum Electron.
{\bf QE-22,} 157--173 (1986).
\bibitem{MD}
L. F. Mollenauer and K. Smith, ``Demonstration of soliton transmission over
more than 4000 km in fiber with loss periodically compensated by Raman
gain,'' Opt. Lett. {\bf 13,} 675--677 (1988).
\bibitem{ME}
L. F. Mollenauer, M. J. Neubelt, S. G. Evangelides, J. P. Gordon, J. R.
Simpson, and L. G. Cohen, ``Experimental study of soliton transmission over
more than 10,000 km in dispersion-shifted fiber,'' Opt. Lett. {\bf 16,}
1203--1205 (1990).
\bibitem{KA}
D. Kr$\rm {\ddot o}$kel, N. J. Halas, G. Giuliani, and D. Grischkowsky,
``Dark-pulse propagation in optical fibers,'' Phys. Rev. Lett. {\bf 60,}
29--32 (1988).
\bibitem{WA}
A. M. Weiner, J. P. Heritage, R. J. Hawkins, R. N. Thurston, E. M.
Kirschner, D. E. Leaird, and W. J. Tomlinson, ``Experimental observation of
the fundamental dark soliton in optical fibers,'' Phys. Rev. Lett. {\bf
61,} 2445--2448 (1988).
\bibitem{WB}
A. M. Weiner, J. P. Heritage, and E. M. Kirschner, ``High-resolution
femtosecond pulse shaping,'' J. Opt. Soc. Am. B {\bf 5,} 1563--1572,
(1988).
\bibitem{ZBA}
W. Zhao and E. Bourkoff, ``Propagation properties of dark solitons,'' Opt.
Lett. {\bf 14,} 703--705 (1989).
\bibitem{ZBB}
W. Zhao and E. Bourkoff, ``Periodic amplification of dark solitons using
stimulated Raman scattering,'' Opt. Lett. {\bf 14,} 808--810 (1989).
\bibitem{ZBC}
W. Zhao and E. Bourkoff, ``Interactions between dark solitons,'' Opt. Lett.
{\bf 14,} 1371--1373 (1989).
\bibitem{ZBD}
W. Zhao and E. Bourkoff, ``Generation of dark solitons under cw background
using waveguide EO modulators,'' Opt. Lett. {\bf 15,} 405--407 (1990).
\bibitem{WC}
A. W. Weiner, R. N. Thurston, W. J. Tomlinson, J. P. Heritage, D. E.
Leaird, E. M. Kirschner, and R. J. Hawkins, ``Temporal and spectral
self-shifts of dark optical solitons,''  Opt. Lett. {\bf 14,} 868--870
(1989).
\bibitem{AB}
G. P. Agrawal, {\it Nonlinear Fiber Optics,} Chapt. 5 (Academic, Boston,
1989).
\bibitem{ZBE}
W. Zhao and E. Bourkoff, ``Dark solitons: generation, propagation,  and
amplification'', {\it OSA Annual Meeting,} Vol. 18 of 1989 OSA Technical
Digest Series (Optical Society of America, Washington, D.C., 1989), p. 185.
\bibitem{ZBF}
W. Zhao and E. Bourkoff, ``Femtosecond pulse propagation in optical fibers:
higher order effects,'' IEEE J. Quantum Electron. {\bf QE-24,} 356--372
(1988).
\bibitem{MG}
F. M. Mitschke and L. F. Mollenauer, ``Discovery of soliton self-frequency
shift,'' Opt. Lett. {\bf 11,} 659--661 (1986);
\bibitem{GD}
J. P. Gordon, ``Theory of soliton self-frequency shift,'' Opt. Lett. {\bf
11,} 662--664 (1986).
\bibitem{WD}
A. W. Weiner, R. N. Thurston, W.J. Tomlinson, J. P. Heritage, D.E. Leaird,
E. M. Kirschner, and R. J. Hawkins, ``Temporal and spectral self-shifts of
dark optical solitons,''  Opt. Lett. {\bf 14,} 868--870 (1989).
\bibitem{SA}
R. H. Stolen, J. P. Gordon, W. J. Tomlinson, and H. A. Haus, ``Raman
response function of silica-core fibers,'' J. Opt. Soc. Am. B {\bf 6,}
1159--1166 (1989).
\bibitem{BB}
K. J. Blow, N. J. Doran, and D. Wood, ``Suppression of the soliton
self-frequency shift by bandwidth-limited amplification,''  J. Opt. Soc.
Am. B {\bf 5,} 1301--1304 (1988).
\bibitem{TA}
K. Tai, A. Hasegawa, and N. Bekki, ``Fission of optical solitons induced by
stimulated Raman effect,'' Opt. Lett. {\bf 13,} 392--394 (1988).
\bibitem{TB}
W. J. Tomlinson, R. J. Hawkins, A. M. Weiner, J. P. Heritage, and R. N.
Thurston, ``Dark optical solitons with finite-width background pulses'', J.
Opt. Soc. Am. B {\bf 6,} 329--334 (1989).
\end{references}




\begin{figure}
\caption{The dark solitons generated by the waveguide
Mach-Zehnder interferometer.  The amplitude of the input cw
light is chosen to be $ a =  \pi /2 $ for (a)-(c).  The
parameter $ \delta $ is (a) 0.8, (b) 0.5, and (c) 0.2.  Part (d) is the case
of optimal operation when $ a =  1.33 $, and $ \delta  =  0.7 $.  In all
cases, the output pulse shapes are plotted as solid curves while
the dashed curves are input pulse shapes.  The pulses shown here are at a
propagation distance of $ z  =  4 $.}
\end{figure}
\begin{figure}
\caption{ Dark solitons under constant gain.  Pulse shapes (solid) when
$\Gamma$=0.05 (a) and 1(b), after certain propagation distance,
$\Gamma$z=1.6, as compared to input pulse shapes (dashed). (c): The pulse
duration, relative to its input,  as a function of $\Gamma z$ at various
$\Gamma$. The solid curve is the adiabatic approximation obtained by
perturbation method. Three values of $\Gamma$ are used: $\Gamma$ = 0.05
(dotted); 0.2 (dash-dotted); and 1 (dashed). Negative $\Gamma$z depicts the
case of loss.}
\end{figure}
\begin{figure}
\caption{ The pulse shapes of amplified dark solitons. (a) $ \delta  =  0.5
$, $ \beta  =  2 ln 1.05 $, $ \Gamma_p L  =  2 $, after 8 amplifying cycles
(solid); (b) $ \delta  =  0.5 $, $ \beta  =  2 ln 1.02 $, $ \Gamma_p L  =
2 $, after 16 amplifying cycles (solid); (c) $ \delta  =  0.5 $, $ \beta  =
2 ln 1.02 $, $ \Gamma_p L  =  0.5 $, after 16 amplifying cycles (solid);
(d) The input pulse is the same as in Fig. 1(c), $ \beta  =  2 ln 1.05 $,
after 8 amplification periods (solid).  The input pulse shapes are plotted
as dashed curves.}
\end{figure}
\begin{figure}
\caption{
(a) The shape of a fundamental dark soliton after a propagation distance of
40 (solid). The normalized time delay $ \tau_d  =  0.01 $.  The dashed
curve is the input pulse shape. (b) The trace of the soliton (solid) as a
function of propagation distance for the situation described by (a). The
dotted curve represents the case for a fundamental bright soliton under
similar conditions.}
\end{figure}
\begin{figure}
\caption{
The shape of a higher-order dark soliton [2 tanh($t$)] after a propagation
distance of 12 for $ \tau_d  =  0.01 $ (solid).  The dotted curve is the
pulse if $ \tau_d  =  0 $, i.e., without ISRS.}
\end{figure}
\begin{figure}
\caption{
(a) The shape of an adiabatically amplified fundamental dark soliton
(solid).  $ \Gamma
 =  0.05 $, $ z  =  16 $, and $ \tau _d  =  0.01 $.  The dotted curve
corresponds to the pulse shape without ISRS;  (b) The trace of the
soliton (solid) for the case of (a).  The dotted curve is a fit as
described by Eq. (11) in the text.}
\end{figure}
\begin{figure}
\caption{Even dark pulses when the input pulse (dashed curve)
is $\kappa_0 |\tanh (t)|$. (a) $\kappa_0=1.56$, and $z=8 $ (solid
curve), (b) $\kappa_0  =  4$ and $ z  =  3.75 $ (solid curve).
In (c), three different input pulses are assumed:
$ 8|\tanh (t)|$ (solid curve),
$ 8 [{1-{\rm exp}(-t^2 /\tau_g^2)}]^{1/2}  $ (dotted curve), and
$ 8 [1- {\rm sech } (t/ \tau_s )]$ (dashed curve).
The propagation distance is $ z  = 8$. }
\end{figure}
\begin{figure}
\caption{
Even dark pulses generated from MZI. The pulse after MZI
is  2 cos$(\pi /2 {\rm sech }^2 t ) $ (dashed curve) and the shape
of secondary dark solitons is shown by the solid curve for $ z  =  4 $.}
\end{figure}
\begin{figure}
\caption{
The loss compensated even dark pulses. The input pulse is  2 cos$(\pi /2
{\rm sech }^2 t ) $ (dotted curve),  the secondary solitons with
fiber losses compensated by stimulated Raman scattering is shown by the solid
curve.  For comparison, the pulse shape without fiber losses is shown by
the dashed curve (same as Fig. 8).  The propagation distance is 4.}
\end{figure}

\begin{table}
\caption{Amplitudes of  Secondary Even Dark Pulses}
\begin{tabular}{cccccr}
&&Input Pulse Shape&&&\\
\cline{2-4}
$\Delta_n$Values&$\kappa_0|{\rm tanh}t|$&$\kappa_0[1-{\rm exp}(-t^2/
{\tau_g}^2)]^{1/2}$&$\kappa_0[1-{\rm sech}(t/\tau_s)]$&Avg.&Range\\
\tableline
$\Delta_1$&0.34&0.30&0.21&0.28&$\pm 25\%$ \\
$\Delta_2$&1.56&1.41&1.26&1.41&$\pm 11\%$ \\
$\Delta_3$&2.47&2.26&2.28&2.34&$\pm 6\%$ \\
$\Delta_4$&3.52&3.25&3.31&3.36&$\pm 6\%$ \\
$\Delta_5$&4.45&4.26&4.42&4.38&$\pm 6\%$ \\
$\Delta_6$&5.52&5.35&5.50&5.50&$\pm 5\%$ \\
\end{tabular}
\end{table}


\end{document}

%%% file josab.tex %%%
