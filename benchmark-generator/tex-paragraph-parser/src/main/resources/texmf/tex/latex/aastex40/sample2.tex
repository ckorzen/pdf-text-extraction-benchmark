% SAMPLE2.TEX -- AASTeX macro package tutorial paper.

% The first item in a LaTeX file must be a \documentstyle command to
% declare the overall style of the paper.  The \documentstyle lines
% that are relevant for the AASTeX macros are shown; one is uncommented out
% so that the file can be processed.

\documentstyle[12pt,aasms4]{article}
%\documentstyle[11pt,aaspp4]{article}
%\documentstyle[aas2pp4]{article}

% The eqsecnum style changes the way equations are numbered.  Normally,
% equations are just numbered sequentially through the entire paper.
% If eqsecnum appears in the \documentstyle command, equation numbers will
% be sequential through each section, and will be formatted "(sec-eqn)",
% where sec is the current section number and eqn is the number of the
% equation within that section.  The eqsecnum option can be used with
% any substyle.

%\documentstyle[11pt,eqsecnum,aaspp4]{article}

% Authors are permitted to use the fonts provided by the American Mathematical
% Society, if they are available to them on their local system.  These fonts
% are not part of the AASTeX macro package or the regular TeX distribution.

%\documentstyle[12pt,amssym,aasms4]{article}

% Here's some slug-line data.  The receipt and acceptance dates will be 
% filled in by the editorial staff with the appropriate dates.  Rules will 
% appear on the title page of the manuscript until these are uncommented 
% out by the editorial staff.

%\received{4 August 1988}
%\accepted{23 September 1988}
%\journalid{337}{15 January 1989}
%\articleid{11}{14}

\slugcomment{Not to appear in Nonlearned J., 45.}

% Authors may supply running head information, if they wish to do so, although
% this may be modified by the editorial offices.  The left head contains a
% list of authors, usually three allowed---otherwise use et al.  The right
% head is a modified title of up to roughly 44 characters.  Running heads
% are not printed.

\lefthead{Djorgovski et al.}
\righthead{Collapsed Cores in Globular Clusters}

% This is the end of the "preamble".  Now we wish to start with the
% real material for the paper, which we indicate with \begin{document}.
% Following the \begin{document} command is the front matter for the
% paper, viz., the title, author and address data, the abstract, and
% any keywords or subject headings that are relevant.

\begin{document}

\title{Collapsed Cores in Globular Clusters,\\
    Gauge-Boson Couplings,\\
    and AAS\TeX\ Macro Sample}

\author{S. Djorgovski\altaffilmark{1,2,3} and Ivan R. King\altaffilmark{1}}
\affil{Astronomy Department, University of California,
    Berkeley, CA 94720}

\author{C. D. Biemesderfer\altaffilmark{4,5}}
\affil{National Optical Astronomy Observatories, Tucson, AZ 85719}

\and

\author{R. J. Hanisch\altaffilmark{5}}
\affil{Space Telescope Science Institute, Baltimore, MD 21218}

% Notice that each of these authors has alternate affiliations, which
% are identified by the \altaffilmark after each name.  The actual alternate
% affiliation information is typeset in footnotes at the bottom of the
% first page, and the text itself is specified in \altaffiltext commands.
% There is a separate \altaffiltext for each alternate affiliation
% indicated above.

\altaffiltext{1}{Visiting Astronomer, Cerro Tololo Inter-American Observatory. 
CTIO is operated by AURA, Inc.\ under contract to the National Science
Foundation.} 
\altaffiltext{2}{Society of Fellows, Harvard University.} 
\altaffiltext{3}{present address: Center for Astrophysics,
    60 Garden Street, Cambridge, MA 02138}
\altaffiltext{4}{Visiting Programmer, Space Telescope Science Institute}
\altaffiltext{5}{Patron, Alonso's Bar and Grill}

% The abstract environment prints out the receipt and acceptance dates
% if they are relevant for the journal style.  For the aasms style, they
% will print out as horizontal rules for the editorial staff to type
% on, so long as the author does not include \received and \accepted
% commands.  This should not be done, since \received and \accepted dates
% are not known to the author.

\begin{abstract}
This is a preliminary report on surface photometry of the major 
fraction of known globular clusters, to see which of them show the signs 
of a collapsed core.
We also show off the results of some recreational mathematics,
and give pause to consider the dangers of the too fertile mind.
\end{abstract}

% The different journals have different requirements for keywords.  The
% keywords.apj file, found on aas.org in the pubs/aastex-misc directory, 
% contains a list of keywords used with the ApJ and Letters.  These are 
% usually assigned by the editor, but authors may include them in their 
% manuscripts if they wish. 

\keywords{clusters: globular, peanut --- bosons: bozos}
%\keywords{globular clusters,peanut clusters,bosons,bozos}

% That's it for the front matter.  On to the main body of the paper.
% We'll only put in tutorial remarks at the beginning of each section
% so you can see entire sections together.

% In the first two sections, you should notice the use of the LaTeX \cite
% command to identify citations.  The citations are tied to the
% reference list via symbolic KEYs.  We have chosen the first three
% characters of the first author's name plus the last two numeral of the
% year of publication.  The corresponding reference has a \bibitem
% command in the reference list below.
%
% Please see the AASTeX manual for a more complete discussion on how to make
% \cite-\bibitem work for you.   

\section{Introduction}

A focal problem today in the dynamics of globular clusters is 
core collapse.  It has been predicted by theory
for decades (\cite{hen61,lyn68,spi85}), but
observation has been less alert to the phenomenon. For many years the 
central brightness peak in M15 (\cite{kin75,new78})
seemed a unique anomaly.  Then Auri\`ere (1982) suggested a central peak 
in NGC 6397, and a limited photographic survey of ours (\cite[Paper I]{djo84})
found three more cases, including NGC 6624, whose
sharp center had often been remarked on (e.g., \cite{can78}).

\section{Observations}

All our observations were short direct exposures with CCD's.  At
Lick Observatory we used a TI 500$\times$500 chip 
and a GEC 575$\times$385, on the 1-m Nickel reflector.  The only
filter available at Lick was red.  At CTIO we used a GEC 575$\times$385, with
$B, V,$ and $R$ filters, and an RCA 512$\times$320, with $U, B, V, R,$ and $I$
filters, on the 1.5-m reflector. In the CTIO observations we tried to
concentrate on the shortest practicable wavelengths; but faintness, reddening,
and poor short-wavelength sensitivity often kept us from observing in $U$ or
even in $B$. All four cameras had scales of the order of 0.4 arcsec/pixel, and
our field sizes were around 3 arcmin. 

The CCD images are unfortunately not always suitable, for very poor 
clusters or for clusters with large cores.  Since the latter are easily 
studied by other means, we augmented our own CCD profiles by collecting
from the literature a number of star-count
profiles (\cite{kin68,pet76,har84,ort85}),
as well as photoelectric profiles (\cite{kin66}) and 
electronographic profiles (\cite{kro84}).
In a few cases we judged normality by eye estimates on one of the Sky
Surveys. 

% Authors may indicate to the editorial staff where they would like 
% figures and tables to be placed in the manuscript.  This is done with
% either the \placefigure{KEY} or \placetable{KEY} commands.  These
% commands require \label{KEY} commands to be placed appropriately with
% corresponding table and figure captions.  When the manuscript is
% printed a short note is printed on the page where the figure or table
% is to go.  These commands are ignored in the aaspp4 and aas2pp4 styles.

\placetable{tbl-3}
\placefigure{fig1}

% In this section, we see the use of the \subsection command to set off
% an independent subsection.  We only have one here; usually there would
% be several.

% We show the use of several of the displayed math environments described
% in the User Guide, and you get a healthy dose of mathematical typesetting
% examples.  Also, observe the use of the LaTeX \label command after the
% \subsection to give a symbolic KEY to the subsection for cross-referencing
% in a \ref command.  LaTeX automatically numbers the sections, equations,
% tables, etc., as it goes, so in general you don't know what number something
% is going to have.  We'll refer to the "hairymath" section a little later.

\section{Helicity Amplitudes}

It has been realized that helicity amplitudes provide a convenient means
for Feynman diagram\footnote{Footnotes can be inserted like this.}
evaluations.  These amplitude-level techniques
are particularly convenient for calculations involving many Feynman
diagrams, where the usual trace techniques for the amplitude
squared becomes unwieldy.  Our calculations use the helicity techniques
developed by other authors (\cite{hag86}); we briefly summarize below.

\placefigure{fig2}

\subsection{Formalism} \label{hairymath}

A tree-level amplitude in $e^+e^-$ collisions can be expressed in
terms of fermion strings of the form
\begin{equation}
\bar v(p_2,\sigma_2)P_{-\tau}\not\!a_1\not\!a_2\cdots
\not\!a_nu(p_1,\sigma_1)\;,
\end{equation}
where $p$ and $\sigma$ label the initial $e^{\pm}$ four-momenta
and helicities $(\sigma = \pm 1)$, $\not\!a_i=a^\mu_i\gamma_\nu$
and $P_\tau=\frac{1}{2}(1+\tau\gamma_5)$ is a chirality projection
operator $(\tau = \pm1)$.  The $a^\mu_i$ may be formed from particle
four-momenta, gauge-boson polarization vectors or fermion strings with
an uncontracted Lorentz index associated with final-state fermions.

% The \notetoeditor{TEXT} command allows the author to communicate some
% information to the copy editor.  This information will appear as a 
% footnote on the printed copy for the aasms4 style file.  Nothing will 
% appear on the printed copy if the aaspp4 or aas2pp4 style file is used.

In the chiral \notetoeditor{This is a note to the copy editor that is
inserted by the author using the {\bf $\backslash$notetoeditor} command.}
representation the $\gamma$ matrices are expressed
in terms of $2\times 2$ Pauli matrices $\sigma$ and the unit matrix 1 as
\begin{mathletters}
\begin{eqnarray}
\gamma^\mu \: & = &
\: \left(
\begin{array}{cc}
0 & \sigma^\mu_+ \\
\sigma^\mu_- & 0
\end{array} \; \; \right)\;,
\;\gamma^5= \left(
\begin{array}{cc}
-1 & \; 0\\
0 & \; 1
\end{array} \; \; \right) \;, \nonumber \\ & & \\
\sigma^\mu_{\pm} \:& = & \: ({\bf 1} ,\pm \sigma)\;, \nonumber
\end{eqnarray}
\end{mathletters}
giving
\begin{equation}
\not\!a= \left(
\begin{array}{cc}
0 & (\not\!a)_+\\
(\not\!a)_- & 0
\end{array}\right),\;(\not\!a)_\pm=a_\mu\sigma^\mu_\pm\;,
\end{equation}
The spinors are expressed in terms of two-component Weyl spinors as
\begin{equation}
u=\left(
\begin{array}{c}
(u)_-\\
(u)_+
\end{array}\right),\;v={\bf (}(v)^\dagger_+{\bf ,} \; (v)^\dagger_-{\bf )}\;.
\eqnum{3A}
\end{equation}
The Weyl spinors are given in terms of helicity eigenstates
$\chi_\lambda(p)$ with $\lambda=\pm1$ by
\begin{eqnarray}
u(p,\lambda)_\pm & = & (E\pm\lambda|{\bf p}|)^{1/2}\chi_\lambda(p)\;,
\nonumber \\ & & \\
v(p,\lambda)_\pm & = & \pm\lambda(E\mp\lambda|{\bf p}|)^{1/2}\chi
_{-\lambda}(p) \nonumber
\end{eqnarray}

% In these sections, we see some additional math-related markup, and we
% have references to one of the tables (occurs later in the document)
% and the "hairymath" section immediately preceding this one.
%
% In the second paragraph, note the use of in-text math ($stuff$) including
% a couple of the miscellaneous symbol commands defined in the AASTeX macro
% package.
%
% This is the last section of the paper, so there is an \acknowledgments
% section at the end of the main body.

\section{Floating material and so forth}

Consider a task that computes profile parameters for a modified
Lorentzian of the form
\begin{equation}
I = \frac{1}{1 + d_{1}^{P (1 + d_{2} )}}
\end{equation}
where
\begin{displaymath}
d_{1} = \sqrt{ \left( \begin{array}{c} \frac{x_{1}}{R_{maj}} 
\end{array} \right) ^{2} + 
\left( \begin{array}{c} \frac{y_{1}}{R_{min}} \end{array} \right) ^{2} }
\end{displaymath}
\begin{displaymath}
d_{2} = \sqrt{ \left( \begin{array}{c} \frac{x_{1}}{P R_{maj}}
\end{array} \right) ^{2} + 
\left( \begin{array}{c} \case{y_{1}}{P R_{min}} \end{array} \right) ^{2} }
\end{displaymath}
\[x_{1} = (x - x_{0}) \cos \Theta + (y - y_{0}) \sin \Theta \]
\[y_{1} = -(x - x_{0}) \sin \Theta + (y - y_{0}) \cos \Theta \]

In these expressions $x_{0}$,$y_{0}$ is the star center, and $\Theta$ is the
angle with the $x$ axis.  Results of this task are shown in table~\ref{tbl-1}.
It is not clear how these sorts of analyses may affect determination of
$M_{\sun}$ and $M_{\earth}$, but the assumption is that the alternate results
should be less than 90\arcdeg\ out of phase with previous values.
We have no observations of \ion{Ca}{2}.
Roughly \slantfrac{4}{5} of the electronically submitted abstracts
for AAS meetings are error-free.

\placetable{tbl-1}
\placetable{tbl-2}
\placefigure{fig3}

\acknowledgments

We are grateful to V. Barger, T. Han, and R. J. N. Phillips for
doing the math in section~\ref{hairymath}.

\appendix
\section{Floating material and so forth}


Consider a task that computes profile parameters for a modified
Lorentzian of the form
\begin{equation}
I = \frac{1}{1 + d_{1}^{P (1 + d_{2} )}}
\end{equation}
where
\begin{mathletters}
\begin{displaymath}
d_{1} = \frac{3}{4} \sqrt{ \left( \begin{array}{c} \frac{x_{1}}{R_{maj}} 
\end{array} \right) ^{2} + 
\left( \begin{array}{c} \frac{y_{1}}{R_{min}} \end{array} \right) ^{2} }
\end{displaymath}
\begin{equation}
d_{2} = \case{3}{4} \sqrt{ \left( \begin{array}{c} \frac{x_{1}}{P R_{maj}}
\end{array} \right) ^{2} + 
\left( \begin{array}{c} \case{y_{1}}{P R_{min}} \end{array} \right) ^{2} }
\end{equation}
\begin{eqnarray}
x_{1} & = & (x - x_{0}) \cos \Theta + (y - y_{0}) \sin \Theta \\ 
y_{1} & = & -(x - x_{0}) \sin \Theta + (y - y_{0}) \cos \Theta 
\end{eqnarray}
\end{mathletters}

For completeness, here is one last equation.
\begin{equation}
e = mc^2
\end{equation}

% That's the end of the main body of the paper.  Now we will have some
% back matter.
%
% Tables are supposed to be submitted one per page, following
% the main body of the text, so before each table we would have a
% \clearpage to force a page break at that point.  There should also
% be a \clearpage after the last table so that it gets forced onto
% its own page, too.
%
% Two options are available to the author for producing tables:  the
% "deluxetable" environment provided by the AASTeX package or the LaTeX
% "table" environment.  The AASTeX "deluxetable" environment is preferred
% by the Production Offices.  Only short tables should be included in the
% body of the text.  If tables extend over a page they should be generated
% using either the apjpt4 or aj_pt4 style file; these styles also use the 
% "deluxetable" environment - but these tables will be produced as
% "camera-ready".
%
% We start with a table using the "deluxetable" environment.
%
% The caption contains only the caption text.  The "Table N." identification
% is generated by the \tablecaption command on its own.  It is necessary to 
% \label tables and figures *after* the caption has been specified because 
% the table/figure number is generated by the caption, not by \begin{whatever}.
% The column headings are specified within a \colhead command and all the
% column headings are included within a single \tablehead command.  The
% \enddata command comes at the end of the data, and the table is closed with 
% an \end{deluxetable} command.  It the table is too wide for the page, \small
% (11pt), \footnotesize (10pt), or \scriptsize (8pt) may be used inside
% the deluxetable environment - the table will still be double-spaced.  For
% even wider tables see the AASTeX guide.
 

\clearpage
 
\begin{deluxetable}{crrrrrrrrrrr}
\footnotesize
\tablecaption{Terribly relevant tabular information. \label{tbl-1}}
\tablewidth{0pt}
\tablehead{
\colhead{Star} & \colhead{Height}   & \colhead{$d_{x}$}   & \colhead{$d_{y}$} & 
\colhead{$n$}  & \colhead{$\chi^2$} & \colhead{$R_{maj}$} & 
\colhead{$R_{min}$}     & \colhead{$P$\tablenotemark{a}}  & 
\colhead{$P R_{maj}$}   & \colhead{$P R_{min}$}           &
\colhead{$\Theta$\tablenotemark{b}}
} 
\startdata
1 &33472.5 &$-$0.1 &0.4  &53 &27.4 &2.065  &1.940 &3.900 &68.3 &116.2 &$-$27.639 \nl
2 &27802.4 &$-$0.3 &$-$0.2 &60 &3.7  &1.628  &1.510 &2.156 &6.8  &7.5 &$-$26.764\nl
3 &29210.6 &0.9  &0.3  &60 &3.4  &1.622  &1.551 &2.159 &6.7  &7.3 &$-$40.272\nl
4 &32733.8 &$-$1.2 &$-$0.5 &41 &54.8 &2.282  &2.156 &4.313 &117.4 &78.2 &$-$35.847\nl
5 & 9607.4 &$-$0.4 &$-$0.4 &60 &1.4  &1.669  &1.574 &2.343 &8.0  &8.9 &$-$33.417\nl
6 &31638.6 &1.6  &0.1  &39 &315.2 & 3.433 &3.075 &7.488 &92.1 &25.3 &$-$12.052\nl
 
\enddata

% Text for table footnotes follows the tabular data and must be inside the
% deluxetable environment.  Note that it is OK to put \ref's in 
% \tablenotetext's.
 
\tablenotetext{a}{Sample footnote for table~\ref{tbl-1} that was generated
with the deluxetable environment}
\tablenotetext{b}{Another sample footnote for table~\ref{tbl-1}}
\tablenotetext{c}{Footnote with no call out}
\tablenotetext{d}{Another footnote with no call out}
\tablenotetext{e}{A further additional footnote with no call out}

 
\end{deluxetable}


% Tabular data can also be aligned within the LaTeX "tabular" environment.  
% Observe that our tabular environment is embedded within a "center" 
% environment, which is in turn inside a "table" environment.  Exercise for 
% the reader:
%
% Why do you think we used the "table*" environment?
%
% We need the table environment for autonumbering and caption generation,
% which is why it is not enough to have a centered tabular.
%
% Within the tabular environment, please note that we use no vertical
% rules, and the only horizontal rule is the \tableline (*not* an \hline)
% which delimits the column headings from the tabular data.  Also note
% that a couple of the column headings require special annotation, i.e.,
% footnotes for tables.  They are marked and tagged with \tablenotemark.
% \tablenotemarks could be placed on individual data entries as well,
% but be careful not to go berserk doing this.

\clearpage

\begin{table*}
\begin{center}
\begin{tabular}{crrrrrrrrrrr}
Star & Height & $d_{x}$ & $d_{y}$ & $n$ & $\chi^2$ & $R_{maj}$ & $R_{min}$ &
\multicolumn{1}{c}{$P$\tablenotemark{t}} & $P R_{maj}$ & $P R_{min}$ & 
\multicolumn{1}{c}{$\Theta$\tablenotemark{u}} \\
\tableline
1 &33472.5 &-0.1 &0.4  &53 &27.4 &2.065  &1.940 &3.900 &68.3 &116.2 &-27.639\\
2 &27802.4 &-0.3 &-0.2 &60 &3.7  &1.628  &1.510 &2.156 &6.8  &7.5 &-26.764\\
3 &29210.6 &0.9  &0.3  &60 &3.4  &1.622  &1.551 &2.159 &6.7  &7.3 &-40.272\\
4 &32733.8 &-1.2\tablenotemark{v} &-0.5 &41 &54.8 &2.282  &2.156 &4.313 &117.4 &78.2 &-35.847\\
5 & 9607.4 &-0.4 &-0.4 &60 &1.4  &1.669\tablenotemark{v}  &1.574 &2.343 &8.0  &8.9 &-33.417\\
6 &31638.6 &1.6  &0.1  &39 &315.2 & 3.433 &3.075 &7.488 &92.1 &25.3 &-12.052\\
\end{tabular}
\end{center}

% Text for table footnotes must follow the tabular environment but must
% be inside the table environment.  Note that it is OK to put \ref's
% in \tablenotetext's.

\tablenotetext{t}{Sample footnote for table~\ref{tbl-2} that was generated with
the \LaTeX\ table environment}
\tablenotetext{v}{Yet another sample footnote for table~\ref{tbl-2}}
\tablenotetext{u}{Another sample footnote for table~\ref{tbl-2}}

\tablenum{1A}
\caption{
More terribly relevant tabular information.  Notice that it is possible, but
not necessarily desirable, to have more than one table on a page where 
each can have associated
independent notes.  We extend the caption with
further pointless drivel to see the effects of lengthy text on
caption formatting. \label{tbl-2}}

\tablecomments{We can also attach a long-ish paragraph of explanatory
material to a table.  This would be done for journals where long
captions are not permitted (usually because the caption is regarded
as the table's title).  A different command would be used if the
paragraph contained a list of references for the table.}

\end{table*}

% Camera-ready tables, produced with either the apjpt4 or aj_pt4 style files,
% can be referenced within a table environment using \dummytable.  This acts
% like a place holder and bumps the table counter.   For this particular
% manuscript, tbl-3 refers to the table in file samp2tbl.tex.

\begin{table}
\dummytable\label{tbl-3}
\end{table}

% This is the last table for this paper (as well as the first), so we
% should follow it with a \clearpage.  In order to force all the floating
% tables out of their buffers and onto vertical page lists, we must use
% \clearpage rather than \newpage. 

\clearpage

% Now comes the reference list.  In this document, we used \cite to call
% out citations, so we must use \bibitem in the reference list, which
% means we use the LaTeX thebibliography environment.  Please note that
% \begin{thebibliography} is followed by a null argument.  If you forget
% this, mayhem ensues, and LaTeX will say "Perhaps a missing item?" when
% you run it.  Do not call us, do not send mail when this happens.  Put
% the silly {} after the \begin{thebibliography}.
%
% Each reference has a \bibitem command to define the citation format
% to be placed in the text (in []) and the symbolic tag used for 
% cross referencing (in {}).
%
% See sample1.tex, or the AASTeX guide, for an alternative to the \cite-
% \bibitem command.

\begin{thebibliography}{}
\bibitem[Auri\`ere 1982]{aur82} Auri\`ere, M.  1982, \aap,
    109, 301
\bibitem[Canizares et al.\ 1978]{can78} Canizares, C. R.,
    Grindlay, J. E., Hiltner, W. A., Liller, W., and 
    McClintock, J. E.  1978, \apj, 224, 39
\bibitem[Djorgovski and King 1984]{djo84} Djorgovski, S.,
    and King, I. R.  1984, \apjl, 277, L49
\bibitem[Hagiwara and Zeppenfeld 1986]{hag86} Hagiwara, K., and
    Zeppenfeld, D.  1986, Nucl.Phys., 274, 1
\bibitem[Harris and van den Bergh 1984]{har84} Harris, W. E.,
    and van den Bergh, S.  1984, \aj, 89, 1816
\bibitem[H\`enon 1961]{hen61} H\'enon, M.  1961, Ann.d'Ap., 24, 369
\bibitem[King 1966]{kin66}  King, I. R.  1966, \aj, 71, 276
\bibitem[King 1975]{kin75}  King, I. R.  1975, Dynamics of
    Stellar Systems, A. Hayli, Dordrecht: Reidel, 1975, 99
\bibitem[King et al.,\ 1968]{kin68}  King, I. R., Hedemann, E.,
    Hodge, S. M., and White, R. E.  1968, \aj, 73, 456
\bibitem[Kron et al.,\ 1984]{kro84} Kron, G. E., Hewitt, A. V.,
    and Wasserman, L. H.  1984, \pasp, 96, 198
\bibitem[Lynden-Bell and Wood 1968]{lyn68} Lynden-Bell, D.,
    and Wood, R.  1968, \mnras, 138, 495
\bibitem[Newell and O'Neil 1978]{new78} Newell, E. B.,
    and O'Neil, E. J.  1978, \apjs, 37, 27
\bibitem[Ortolani et al.,\ 1985]{ort85} Ortolani, S., Rosino, L.,
    and Sandage, A.  1985, \aj, 90, 473
\bibitem[Peterson 1976]{pet76} Peterson, C. J.  1976, \aj, 81, 617
\bibitem[Spitzer 1985]{spi85} Spitzer, L.  1985, Dynamics of
    Star Clusters, J. Goodman and P. Hut, Dordrecht: Reidel, 109
\end{thebibliography}

% And finally, we must deal with the figures.  There are three figures
% associated with this manuscript; two figures are Encapsulated
% PostScript (EPS) files.  The third figure is a grey scale figure that does
% not exist in EPS form.
%
% Authors have three options for including figure information within a 
% manuscript.  Not all the options may be acceptable by the target Journal - be
% sure to look at the appropriate submission instructions, electronic or 
% otherwise.
%
% Option 1.  Using this option, only the figure captions are included in the
% main body of the manuscript.  The figure captions must start on a new page.
% The captions are generated with the \figcaption[]{} command: the first 
% argument is optional, if you put something in there, put the name of the 
% EPS file that goes with the caption; the second argument is the figure 
% caption itself, and may include a \label command.  The \figcaption command
% generates the figure numbers.  This option is acceptable for all manuscript
% submissions.

\clearpage

\figcaption[sgi9259.eps]{This is the first figure and it uses sgi9259.eps as
its EPS figure file. \label{fig1}}

\figcaption{This figure has no associated EPS file, so the optional 
parameter is omitted. \label{fig3}}

\figcaption[sgi9279.eps]{This is an example of a long figure caption that
must be set as a paragraph.  The processor has to buffer the text of the
caption, so it is good not to be too wordy, but that would make for
poor communication as well. \label{fig2}}

\end{document}

% Option 2.  The figure captions are printed on a caption page(s) as in 
% option 1.  The figures available as EPS files are then printed at the
% end of the document, one figure per page, using the \plotone command.
% If you wish to process this option then simply comment out the \end{document}
% just above these five lines. 

\clearpage

\plotone{sgi9259.eps}

\clearpage

\plotone{sgi9279.eps}


\end{document}

% Option 3. Figures and figure captions are included within figure 
% environments within the body of the manuscript.  In our examples the 
% \plotone command is placed in the figure environment along with the
% figure caption.  The \caption command can also include a \label command.
% Each figure and its caption are printed on the same page.
%
% The \caption command in the figure environment works like the one in the
% table environment (it's the same one, actually), except that this one
% produces identification text that reads "Figure N."
%
% If you wish to see this option then you must comment out all of the 
% \figcaption, \plotone, and \end{document} commands above.

\clearpage

\begin{figure}
\plotone{sgi9259.eps}
\caption{This is the first figure and it uses sgi9259.eps as
its EPS figure file. \label{fig1}}
\end{figure}

\begin{figure}
\caption{This figure has no associated EPS file, so only the caption
is printed. \label{fig2}}
\end{figure}

% The \plotone and \plottwo commands scale the plot(s) in both dimensions
% so that the horizontal dimension fits in the body of the text.  The
% \plotfiddle command will override any automatic scaling, but often
% requires additional "fiddling" to get the plot to fit on the page.
% The \epsscale command allows the author to simply change the scaling
% of the plot in place, without the additional "fiddling" required by 
% \plotfiddle.

\begin{figure}
\epsscale{.6}
\plotone{sgi9259.eps}
\caption{This is an example of a long figure caption that must be set as
a paragraph.  The processor has to buffer the text of the
caption, so it is good not to be too wordy, but that would make for
poor communication as well. \label{fig3}}
\end{figure}



% That's all, folks.
%
% The technique of segregating major semantic components of the document
% within "environments" is a very good one, but you as an author have to
% come up with a way of making sure each \begin{whatzit} has a corresponding
% \end{whatzit}.  If you miss one, LaTeX will probably complain a great
% deal during the composition of the document.  Occasionally, you get away
% with it right up to the \end{document}, in which case, you will see
% "\begin{whatzit} ended by \end{document}".

\end{document}
