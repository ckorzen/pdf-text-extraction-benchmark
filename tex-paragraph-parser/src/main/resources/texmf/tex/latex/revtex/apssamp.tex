%%% ======================================================================
%%%  @LaTeX-file{
%%%     filename        = "apssamp.tex",
%%%     version         = "3.0",
%%%     date            = "November 10, 1992",
%%%     ISO-date        = "1992.11.10",
%%%     time            = "15:41:54.18 EST",
%%%     author          = "American Physical Society",
%%%     contact         = "Christopher B. Hamlin",
%%%     address         = "APS Publications Liaison Office
%%%                        500 Sunnyside Blvd.
%%%                        Woodbury, NY 11797",
%%%     telephone       = "(516) 576-2390",
%%%     FAX             = "(516) 349-7817",
%%%     email           = "mis@aps.org (Internet)",
%%%     supported       = "yes",
%%%     archived        = "pinet.aip.org/pub/revtex,
%%%                        Niord.SHSU.edu:[FILESERV.REVTEX]",
%%%     keywords        = "REVTeX, version 3.0, sample,
%%%                        American Physical Society",
%%%     codetable       = "ISO/ASCII",
%%%     checksum        = "58025 643 2971 22909",
%%%     docstring       = "This is the sample of usage for REVTeX 3.0.
%%%
%%%                        The checksum field above contains a CRC-16
%%%                        checksum as the first value, followed by the
%%%                        equivalent of the standard UNIX wc (word
%%%                        count) utility output of lines, words, and
%%%                        characters.  This is produced by Robert
%%%                        Solovay's checksum utility."
%%% }
%%% ======================================================================
% ****** Start of file apssamp.tex ******
%
%   This file is part of the APS files in the REVTeX 3.0 distribution.
%   Version 3.0 of REVTeX, November 10, 1992.
%
%   Copyright (c) 1992 The American Physical Society.
%
%   See the REVTeX 3.0 README file for restrictions and more information.
%
%
%
% \documentstyle[preprint,eqsecnum,aps]{revtex}
\documentstyle[eqsecnum,aps]{revtex}
\def\btt#1{{\tt$\backslash$#1}}
\begin{document}
\draft
\preprint{HEP/123-qed}
\title{Title of manuscript:\\
Force line breaks with $\backslash\backslash$
}
\author{A. A.  Author and B. B. Author\cite{byline}\\
Lines break automatically or can be forced
with $\backslash\backslash$}
\address{
Authors' institution and/or address\\
This line break forced with $\backslash\backslash$
}
\author{C. C. Author}
\address{
Second author institution and/or address\\
This line break forced with $\backslash\backslash$
}
\date{\today}
\maketitle
\begin{abstract}
The author will not know the received date when the compuscript
is first submitted; production will insert this. Every
article includes an abstract.  The abstract is a concise
summary of the work covered at length
in the main body of the article.
It is used for secondary publications and for information retrieval
purposes. Valid PACS numbers should be entered after the abstract
is finished, using the \verb+\pacs{#1}+ command.
\end{abstract}
\pacs{Valid PACS appear here.
{\tt$\backslash$\string pacs\{\}} should always be input,
even if empty.}

\narrowtext

\section{First-level heading:\protect\\ The line break was forced via
$\backslash\backslash$}
\label{sec:level1}

Here is the first sentence in Sec.\ \ref{sec:level1}, demonstrating
section cross-referencing. Note that this sample file was run
with the eqsecnum option selected.
Here is an openface one: $\openone$.

This file (apssamp.tex) contains comments marking the start/end of the
pages of galley-style output. This should make it easier to compare the
output to the input file.

\subsection{Second-level heading:\protect\\ The line break was forced via
$\backslash\backslash$}
\label{sec:level2}

Here is the first sentence in Sec.\ \ref{sec:level2}, demonstrating
section cross-referencing.
The command \btt{narrowtext}
will make the text this width. The command \btt{widetext}
will make the text the width of the full page, as on page \pageref{wideeq}.
A blank input line tells \TeX\ that a new paragraph begins.

The width-changing commands only take effect in galley style (the default
style). Preprint style gives output of a constant width.

This file may be run in both preprint and galley styles. Preprint
format is used for submission purposes. Galley format is used in production
to produce final output.

When commands are referred to in this example file, they are always shown
with their mandatory arguments, using normal \TeX{} format. In this format,
\verb+#1+, \verb+#2+, etc. stand for mandatory
author-supplied arguments to commands.
For example, in
\verb+\section{#1}+ the \verb+#1+ stands for the text of the author's section
heading, and in \verb+\title{#1}+ the \verb+#1+ stands for the
title of the paper.

%***
%***  E n d   o f   p a g e   1   o f   g a l l e y - m o d e   o u t p u t
%***

Reference citations in text use the command \verb+\cite{#1}+.
\verb+#1+ may contain letters and numbers.
In the reference section of this paper
each reference is ``tagged'' by the \verb+\bibitem{#1}+ command.
\verb+#1+ should be {\em identical\/} in both commands.
The proper form for citing in text is
\verb+\cite{#1}+,
and the result is shown here \cite{smith82,jones78}.
We will cite  other people \cite{smith82,jonessmith80}
and journals here. We also cite other people again (Refs.\
\onlinecite{smith82} and \onlinecite{jonessmith80}).
It is worth mentioning that REV\TeX{} ``collapses'' lists
of reference numbers where possible.  We now cite
everyone together \cite{smith82,jones78,jonessmith80}, and once again
(Refs.\ \onlinecite{smith82,jones78,jonessmith80}).

When the {\tt prb} option is used, the command \verb+\onlinecite{#1}+ will
put the reference citations on-line. It was used in the preceding paragraph.
Note that the location of citations must be adjusted to the reference style:
the superscript references in {\tt prb} style must appear after punctuation;
other styles must appear before any punctuation. This sample was written
for the regular (non-{\tt prb}) citation style, but invoking the
{\tt prb} option will show the results of  the command \verb+\onlinecite{#1}+
in the preceding paragraph.



\section{Displayed equations}
\subsection{Another second-level heading}
\subsubsection{Third-level heading:\protect\\ The line break was forced via
$\backslash\backslash$}
\label{sec:level3}

Here is the first sentence in Sec.\ \ref{sec:level3}, demonstrating
section cross-referencing.
In \LaTeX\ there are many different ways to display equations, and a
few preferred ways are noted below.
Displayed math will center by default.

\paragraph{Fourth-level heading: Single-line equations.}
Below we have single-line equations with numbers; this is
the most common type of equation in {\it Physical Review\/}:
\begin{equation}
\chi_+(p)\alt{\bf [}2|{\bf p}|(|{\bf p}|+p_z){\bf ]}^{-1/2}
\left(
\begin{array}{c}
|{\bf p}|+p_z\\
px+ip_y
\end{array}\right)\;,
\end{equation}
\begin{equation}
\left\{\openone234567890abc123\alpha\beta\gamma\delta%
1234556\alpha\beta{1\sum^{a}_{b}\over A^2}\right\}\label{one}.
\end{equation}
Note the open one in Eq.\ (\ref{one}).

Not all numbered equations will fit
within a narrow column this way. The equation number will move down
automatically if it cannot fit on the same line with a one-line equation:
\begin{equation}
\left\{ab12345678abc123456abcdef\alpha\beta\gamma\delta%
1234556\alpha\beta{1\sum^{a}_{b}\over A^2}\right\}.
\end{equation}

When the \verb+\label{#1}+ command is used [cf. input
for Eq. (\ref{one})],
the equation can be referred to in text without your knowing the
%***
%***  E n d   o f   p a g e   2   o f   g a l l e y - m o d e   o u t p u t
%***
equation number that \TeX\ will assign to it. Just use
\verb+\ref{#1}+, where \verb+#1+ is the same name that you used in the
\verb+\label{#1}+ command.

The \verb+\FL+ and  \verb+\FR+ commands will set displayed math flush
left and flush right, respectively. Just insert the \verb+\FL+ or  \verb+\FR+
command before the displayed math begins. For example, here is an equation
flushed left:
\FL
\begin{equation}
\left\{ab12345678bcdef\alpha\beta\gamma\delta%
1234556\alpha\beta{1\sum^{a}_{b}\over A^2}\right\}.
\end{equation}
You shouldn't need \verb+\FL+ and  \verb+\FR+  very often.


If you have a single-line equation that you don't want
numbered, you can use the \btt{[}, \btt{]} format:
\[g^+g^+ \rightarrow g^+g^+g^+g^+ \dots ~,~~q^+q^+\rightarrow
q^+g^+g^+ \dots ~. \]

\subsubsection{Multiline equations}

Multiline equations are obtained by using the
\btt{begin$\{$eqnarray$\}$}, \btt{end$\{$eqnarray$\}$} format.
Use the \btt{nonumber}
command at the end of each line where you do not want a number:
\begin{eqnarray}
{\cal M}=&&ig_Z^2(4E_1E_2)^{1/2}(l_i^2)^{-1}
\delta_{\sigma_1,-\sigma_2}
(g_{\sigma_2}^e)^2\chi_{-\sigma_2}(p_2)\nonumber\\
&&\times
[\epsilon_jl_i\epsilon_i]_{\sigma_1}\chi_{\sigma_1}(p_1),
\end{eqnarray}
\begin{eqnarray}
\sum \vert M^{\rm viol}_g \vert ^2&=&g^{2n-4}_S(Q^2)~N^{n-2}
        (N^2-1)\nonumber \\
 & &\times \left( \sum_{i<j}\right)
  \sum_{\rm perm}
 {1 \over S_{12}}
 {1 \over S_{12}}\sum_\tau c^f_\tau~.
\end{eqnarray}
{\bf Note:} do not use \verb+\label{#1}+ on a line of a multiline
equation if \verb+\nonumber+ is also used on that line. Incorrect
cross-referencing will result.

If you wish to set a multiline equation without {\em any\/} equation numbers,
you can use the \verb+\begin{eqnarray*}+,
\verb+\end{eqnarray*}+ format:
\begin{eqnarray*}
\sum \vert M^{\rm viol}_g \vert ^2&=&g^{2n-4}_S(Q^2)~N^{n-2}
        (N^2-1)\\
 & &\times \left( \sum_{i<j}\right)
 \left( \sum_{\rm perm}
 {1 \over S_{12}S_{23}S_{n1}}\right)
 {1 \over S_{12}}~.
\end{eqnarray*}
To obtain numbers not normally produced by the automatic numbering,
use the \verb+\eqnum{#1}+ command, where \verb+#1+ is the desired
equation number. For example, to get an equation number of
(\ref{eq:mynum}),
\begin{equation}
g^+g^+ \rightarrow g^+g^+g^+g^+ \dots ~,~~q^+q^+\rightarrow
q^+g^+g^+ \dots ~. \eqnum{2.6$'$}\label{eq:mynum}
\end{equation}

{\it A few notes on} \verb=\eqnum{#1}=.
The \verb+\eqnum{#1}+ must come before the \verb+\label{#1}+, if any.
The numbering set with \verb+\eqnum{#1}+ is {\it transparent} to the
automatic numbering in REV\TeX{}; therefore,
you must know the number ahead of time, and {\it must\/} make
%***
%***  E n d   o f   p a g e   3   o f   g a l l e y - m o d e   o u t p u t
%***
sure that the number set with \verb+\eqnum{#1}+ stays in step
with the automatic numbering.
\verb+\eqnum{#1}+ works with both single-line and multiline equations.
You could, if you wished, do all the numbering in a paper
manually with \verb+\eqnum{#1}+.

Enclosing single-line and multiline equations in
\verb+\begin{mathletters}+ and \verb+\end{mathletters}+ will produce
a set of equations that are ``numbered'' with letters, as shown
in Eqs.\ (\ref{mlett:1}) and (\ref{mlett:2}) below:
\begin{mathletters}
\label{generallabel}
\begin{equation}
\left\{abc123456abcdef\alpha\beta\gamma\delta%
1234556\alpha\beta{1\sum^{a}_{b}\over A^2}\right\},\label{mlett:1}
\end{equation}
\begin{eqnarray}
{\cal M}=&&ig_Z^2(4E_1E_2)^{1/2}(l_i^2)^{-1}
(g_{\sigma_2}^e)^2\chi_{-\sigma_2}(p_2)\nonumber\\
&&\times
[\epsilon_i]_{\sigma_1}\chi_{\sigma_1}(p_1).\label{mlett:2}
\end{eqnarray}
\end{mathletters}
If you use a \verb+\label{#1}+ command right after the
\verb+\begin{mathletters}+, then \verb+\ref{#1}+ can be used to reference
all the equations in a mathletters environment. For example, the equations
in the preceding mathletters environment were Eqs.\ (\ref{generallabel}).

\subsubsection{Wide equations}
The equation that follows is set in a wide format, i.e., it
spans across the full page.  The wide format is reserved for
long equations that cannot be easily broken into four lines or less:
\widetext
\begin{equation}
{\cal R}^{(\rm d)}=
 g_{\sigma_2}^e\left({[\Gamma^Z(3,21)]_{\sigma_1}\over
Q_{12}^2-M_W^2}+{[\Gamma^Z(13,2)]_{\sigma_1}\over Q_{13}^2-M_W^2}
\right) +x_WQ_e\left({[\Gamma^\gamma(3,21)]_{\sigma_1}\over
Q_{12}^2-M_W^2}+{[\Gamma^\gamma(13,2)]_{\sigma_1}
\over Q_{13}^2-M_W^2} \right)\;. \label{wideeq}
\end{equation}
This is typed so you can see that the output
is in wide format.  (Since
there is no input line between \btt{end$\{$equation$\}$}
and this paragraph,
there is no paragraph indent for this paragraph.) We also have
\begin{equation}
{\cal R}^{(f)}=-g^3\delta_{\sigma_1,\sigma_2}
\left( {g^e_{\sigma_2}D_Z\over\cos\theta_W}-Q_eD_\gamma
\cos\theta_W \right)
\left( {[\epsilon_3]_{\sigma_1}\over
Q^2_{12}-M^2_W/\xi}\epsilon_1\cdot\epsilon_2+
{[\epsilon_2]_{\sigma_1}\over
Q^2_{13}-M^2_W/\xi}\epsilon_1\cdot\epsilon_3 \right)\;.
\end{equation}

\narrowtext
\section{Cross-referencing}
REV\TeX{} will automatically number sections, equations,
figure captions, and tables. In order to
reference them in text, use the \verb+\label{#1}+ and \verb+\ref{#1}+
commands.

The \verb+\label{#1}+ command appears following a section heading;
within an equation; or within a figure
or table environment, inside of or following the caption.
The \verb+\ref{#1}+ command appears in text
where citation is to occur.  We will refer to the first
figure (Fig.~\ref{autonum}) here.
We can refer to the ``late figure'' also (Fig.~\ref{latefigure}).

References to figures: Fig.~\ref{autonum}, Fig.~\ref{latefigure},
Fig.~\ref{reduced}, and Fig.~\ref{fig4}.

References to tables:
Table \ref{table1},
Table \ref{table2},
Table \ref{table3},
Table \ref{table4},
Table \ref{latetable}, and
Table \ref{table6}.

{\it Physical Review}
style requires that the initial citation of figures or tables
be in numerical order in text, so don't cite Fig.~\ref{reduced}
until you've cited Fig.~\ref{latefigure}.
See {\it Style and Notation
Guide}.


%***
%***  E n d   o f   p a g e   4   o f   g a l l e y - m o d e   o u t p u t
%***


\acknowledgments

We wish to acknowledge the support of the author community in using
REV\TeX{}, offering suggestions and encouragement, testing new versions,
$\ldots$ .

If a section does not have a number (like the Acknowledgments
section), use the so-called ``star version'' of the command. That is,
insert a star between the command and its arguments:
\verb+\section*{#1}+, \verb+\subsection*{#1}+, etc.
For the Acknowledgments section you can also use the command
\verb+\acknowledgments+ to produce the heading.

\appendix
\section{Appendixes}

To start the appendixes, you should use the \verb+\appendix+ command.
This signals that all following section commands refer to appendixes instead
of regular sections.
Therefore, the \verb+\appendix+ command should be used only once---to setup
the section commands to act as appendixes. Thereafter normal section commands
are used.
The heading for a section can be left empty. For example,
\begin{verbatim}
\appendix
\section{}
\end{verbatim}
will produce an appendix heading that says ``APPENDIX A'' and
\begin{verbatim}
\appendix
\section{Background}
\end{verbatim}
will produce an appendix heading that says ``APPENDIX A: BACKGROUND'' (note
that the colon is set automatically).

 If there is only
one appendix, then the letter ``A'' should not appear. This is suppressed by
using the star version of the section command (\verb+\section*{#1}+).


\section{A little more on appendixes}

Observe that this appendix was started by using
\begin{verbatim}
\section{A little more on appendixes}
\end{verbatim}

Note the equation number in an appendix:
\begin{equation} E=mc^2. \end{equation}

\subsection{A subsection in an appendix}
\label{app:subsec}

You can use a subsection or subsubsection in an
appendix. Note the numbering: we are now in Appendix \ref{app:subsec}.


%***
%***  E n d   o f   p a g e   5   o f   g a l l e y - m o d e   o u t p u t
%***



Note the equation numbers in this appendix, produced
with the mathletters environment:
\begin{mathletters}
\begin{equation} E=mc, \label{appa}\end{equation}
\begin{equation} E=mc^2, \label{appb}\end{equation}
\begin{equation} E\agt mc^3. \label{appc}\end{equation}
\end{mathletters}
They turn out to be Eqs.\ (\ref{appa}), (\ref{appb}),
and (\ref{appc}).

\begin{references}
\bibitem[*]{byline}  Also at Physics Department, XYZ University.
\bibitem{smith82}A. Smith  and B. Doe, J. Chem.\  Phys.\ {\bf
76}, 4056 (1982).
\bibitem{jones78}C. Jones, J. Chem.\ Phys. {\bf 68}, 5298 (1978).
\bibitem{jonessmith80}C. Jones and A. Smith,
J. Chem.\ Phys.\ {\bf 72,}
3416 (1980); {\bf 73,} 5168 (1980); {\bf72,} 4009 (1980).
\end{references}



\begin{figure}
\caption{A figure caption.  The figure captions are automatically
numbered.}
\label{autonum}
\end{figure}

\begin{figure}
\caption{The ``late figure.'' This figure was inserted when the paper
was finished.  Since the figures are automatically numbered,
no renumbering in text was necessary. All that
needed to be done was to type the caption in the
proper place and cite the figure in text.\label{latefigure}}
\end{figure}

\begin{figure}
\caption{A figure caption. Figures will be reduced to an appropriate
size by  the production
staff upon receipt.}
\label{reduced}
\end{figure}

\begin{figure}
\caption{A figure caption.  The labels you give tables and figures
can be descriptive (as that of Fig.\ \protect\ref{autonum}, which has
a  \btt{label}$\{${\tt autonum}$\}$) or they can
reflect their numerical
order, as that of this figure
(\btt{label}$\{${\tt fig4}$\}$).\label{fig4}}
\end{figure}


\begin{table}
\caption{This is a narrow table, which
occupies the width of a narrow
column. The table captions are automatically numbered.
This table shows left-aligned, centered, and right-aligned columns. It also
shows one of two possible methods of setting tablenotes (footnotes within
tables). In this table the tablenotes are numbered and set automatically.
All the author need do is use \btt{tablenote$\{$\#1$\}$} to set a tablenote
mark and its text.
\label{table1}}
\begin{tabular}{lcr}
One\tablenote{Note a.}&Two\tablenote{Note b.}&Three\\
\tableline
one&two&three\\
one&two&three\\
\end{tabular}
\end{table}


%***
%***  E n d   o f   p a g e   6   o f   g a l l e y - m o d e   o u t p u t
%***


\mediumtext
\begin{table}
\caption{This is a table of medium width.
This table shows tablenotes where the author has numbered the tablenotes
by hand. In this approach, \btt{tablenotemark[\#1]} is used to produce the
tablenote mark. {\tt\#1} is a numeric value. Each time the same value
for {\tt\#1} is used,
the same mark is produced in the table. After the end of the tabular
environment,  \btt{tablenotemark[\#1]$\tt\{$\#2$\tt\}$}  commands are used:
 {\tt\#1} represents the same numbers used in \btt{tablenotemark[\#1]}
and {\tt\#2} represents the text of the tablenote. Using these two commands
will allow the author to number tablenotes by hand.
 Inspecting the input for this table should clarify any questions.
\label{table2}}
\begin{tabular}{cccccccc}
 &$r_c$ (\AA)&$r_0$ (\AA)&$\kappa r_0$&
 &$r_c$ (\AA) &$r_0$ (\AA)&$\kappa r_0$\\
\tableline
Cu& 0.800 & 14.10 & 2.550 &Sn\tablenotemark[1]
& 0.680 & 1.870 & 3.700 \\
Ag& 0.990 & 15.90 & 2.710 &Pb\tablenotemark[2]
& 0.450 & 1.930 & 3.760 \\
Au& 1.150 & 15.90 & 2.710 &Ca\tablenotemark[3]
& 0.750 & 2.170 & 3.560 \\
Mg& 0.490 & 17.60 & 3.200 &Sr\tablenotemark[4]
& 0.900 & 2.370 & 3.720 \\
Zn& 0.300 & 15.20 & 2.970 &Li\tablenotemark[2]
& 0.380 & 1.730 & 2.830 \\
Cd& 0.530 & 17.10 & 3.160 &Na\tablenotemark[5]
& 0.760 & 2.110 & 3.120 \\
Hg& 0.550 & 17.80 & 3.220 &K\tablenotemark[5]
&  1.120 & 2.620 & 3.480 \\
Al& 0.230 & 15.80 & 3.240 &Rb\tablenotemark[3]
& 1.330 & 2.800 & 3.590 \\
Ga& 0.310 & 16.70 & 3.330 &Cs\tablenotemark[4]
& 1.420 & 3.030 & 3.740 \\
In& 0.460 & 18.40 & 3.500 &Ba\tablenotemark[5]
& 0.960 & 2.460 & 3.780 \\
Tl& 0.480 & 18.90 & 3.550 & & & & \\
\end{tabular}
\tablenotetext[1]{Here's the first, from Ref.\ \cite{smith82}.}
\tablenotetext[2]{Here's the second.}
\tablenotetext[3]{Here's the third.}
\tablenotetext[4]{Here's the fourth.}
\tablenotetext[5]{And etc.}
\end{table}

\widetext
\begin{table}
\caption{A wide table.  Two alternative occupations of special
positions
by KMnCL$_3$ ions in the two space groups $D_{4h}^1$ and $D_{4h}^1$.
  For a special value of the $x$ and $y$ parameters, a
set of special positions may
split into two sets of special positions of higher symmetry.}
\begin{tabular}{ccccc}
 &\multicolumn{2}{c}{$D_{4h}^1$}&\multicolumn{2}{c}{$D_{4h}^5$}\\
 Ion&1st alternative&2nd alternative&lst alternative
&2nd alternative\\ \tableline
 K&$(2e)+(2f)$&$(4i)$ &$(2c)+(2d)$&$(4f)$ \\
 Mn&$(2g)$\tablenote{The $z$ parameter of these positions is $z\sim\case 1/4$.}
 &$(a)+(b)+(c)+(d)$&$(4e)$&$(2a)+(2b)$\\
 Cl&$(a)+(b)+(c)+(d)$&$(2g)$\tablenotemark[1]
 &$(4e)^{\rm a}$\\
 He&$(8r)^{\rm a}$&$(4j)^{\rm a}$&$(4g)^{\rm a}$\\
 Ag& &$(4k)^{\rm a}$& &$(4h)^{\rm a}$\\
 \end{tabular}
 \label{table3}
 \end{table}

\begin{table}
\caption{Another wide table. Numbers in columns Three--Five have been aligned
by using the ``d'' column specifier. Non-numeric entries (those entries
without a ``.'') are centered in ``d'' columns.}
\begin{tabular}{ccddd}
One&Two&Three&Four&Five\\
\tableline
one&two&three&four&five\\
He&2& 2.77234 & 45672. & 0.69 \\
C\tablenote{Some tables require footnotes.}
  &C\tablenote{Some tables need more than one footnote.}
  & 12537.64 & 37.66345 & 86.37 \\
\end{tabular}
\label{table4}
\end{table}


%***
%***  E n d   o f   p a g e   7   o f   g a l l e y - m o d e   o u t p u t
%***

\narrowtext

\begin{table}
\caption{A ``late table.''  This table was added after most of the
paper had been completed. Since the tables are
automatically numbered, no renumbering in text was necessary. This
table
was added to show the use of the the ``d'' column and the
@ specifier for lining things up. The ``d'' column is useful for simpler
columns of numerical data, but it may be necessary to use multiple columns
and the @ specifier for more complex alignments.}
\begin{tabular}{dr@{}l@{${}\pm{}$}r@{}l}
%% NOTE, multicolumn NEEDED in next line
\multicolumn{1}{c}{Align by .}&
  \multicolumn{4}{c}{Multiple alignments}\\
\tableline
23.890\,12        &23&.890\,12&    0&.002\\
12\,323.          &123\,223&&    344& \\
0.834\,390\,12    &80&.80&        45&.3416\\
\end{tabular}
\label{latetable}
\end{table}

\narrowtext

\begin{table}
\caption{The Poisson ratio defined as the ratio of lateral
contraction to longitudinal expansion for uniaxial stress.
 Experimental values are given for comparison.}
\begin{tabular}{dddddd}
  &\multicolumn{2}{c}{$\sigma$}& &\multicolumn{2}{c}{$\sigma$}\\
 &Predicted&Observed$^{\rm a}$& &Predicted&Observed$^{\rm a}$\\
\tableline
 .48&  0.48 & 0.36 &Al& 0.47 &0.33 \\
0.&  0.48 &0.37 &Tl& 0.47 & 0.35\\
 .48&  0.48 & 0.36&Sn& 0.46 &0.33\\
0.0&  0.47 &0.35&Pb& 0.46 &\multicolumn{1}{c}{0.40--0.45}\\
Zn&  0.47 & 0.25 &Pb& 0.49 & 0.43 \\
 & & &K& 0.49 & 0.44 \\
\end{tabular}
\label{table6}
\end{table}
\end{document}


%
% ****** End of file apssamp.tex ******
