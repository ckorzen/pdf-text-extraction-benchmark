% SAMPLE1.TEX -- AASTeX sample paper with minimal markup.

\documentstyle[12pt,aasms4]{article}

\begin{document}

\title{AAS\TeX\ Sample Papers. I. The Minimalist Approach}
\author{M. Headroom}
\affil{Industrial Metaphysics, Inc., Alluvia, HG 67555}
\authoremail{headroom@nowhere.edu}

\begin{abstract}
This example illustrates how to use the AAS\TeX\ markup in a
way that is as unobtrusive as possible while still identifying
all the important structural parts of the paper.
The most salient thing to observe is that, apart from the document
style declaration, no formatting instructions are given in the file.
\end{abstract}

\keywords{Brevity --- models}

\section{Introduction}

Reader, this is my paper.  Paper, this is our reader.

\section{Observations}

The observations upon which this paper is based were taken on Wednesday
while I was grocery shopping.  I needed a half-gallon of milk, chips and
salsa, and a bag of kitty litter \markcite{berl1837} (Berlioz 1837). 
Calibration data were taken on Friday
when I went back for a six-pack of beer.

\section{Discussion}

Grocery stores \notetoeditor{Of all the stores in Tucson my favorite is never
open.} seem to be inordinately crowded on Wednesdays and Fridays
\markcite{head1988} (Headroom 1988).  The increase in Friday-shopper density can
be understood by assuming that many people get paid on Fridays,
and by recognizing that such people often do not work on Saturdays and
Sundays and can be assumed to be ``stocking up'' for the weekend.

The Wednesday peak is harder to explain, but may be related to the
delivery of fresh produce on Tuesday nights.  This interpretation
depends on the assumption that many people 
eat sensibly and therefore find fresh produce attractive.

\acknowledgments

My cats, Hal and Yoda, provided motivation for the initiation of this study.

\begin{references}
\reference{berl1837} Berlioz, H. 1837, Grande Messe du Morts (Paris: Durand)
\reference{head1988} Headroom, M. 1988, \apj, 278, 356
\end{references}

\end{document}
