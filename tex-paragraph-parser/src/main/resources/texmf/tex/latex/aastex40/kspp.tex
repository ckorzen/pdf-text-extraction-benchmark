% KSPP.TEX -- K.Sellgren's proposal about preprints for AAS Pubs Board.
%
% I have made a few modifications to the file so that the markup reads
% as true preprint markup.  Kris had to prepare this with a couple
% items handled explicitly; I have added the commands for these.

% The style file names were updated for compatibility with v4.0.  And
% an argument was added to each \reference command so that the file 
% would latex properly under v4.0.  The \slugcomment was moved to put it
% in the preamble where it should be.  jb 4/20/95

\documentstyle[12pt,aaspp4]{article}
%\documentstyle[12pt,aasms4]{article}
%\tighten

\slugcomment{\shortstack[r]{
Submitted to the AAS Publications Board, 1992 June 1\\
Revised, 1992 November 17}}
%\hbox to 6.5truein {\hfil Submitted to the AAS Publications Board, 1992 June 1}
%\hbox to 6.5truein {\hfil Revised, 1992 November 17}

\begin{document}

\title
{A Proposed Format for the AAS/WGAS Latex Macros Preprint Style}

\author
{K. Sellgren\altaffilmark{1}}
%{K. Sellgren$^ {1}$}

\affil
{Department of Astronomy, Ohio State University, Columbus, OH  43210}

\altaffiltext{1}
{Alfred P. Sloan Foundation Fellow.}

%\vskip 24pt
%\clearpage

%\centerline
%{\bf Abstract}
\begin{abstract}
This is a sample preprint for defining the AAS/WGAS Latex Macros preprint 
style.  Results are presented for two surveys of preprint formats preferred 
by astronomers.
\end{abstract}

\keywords{publications: preprint --- publications: or perish --- 
publications: sample --- AAS: Pub Board --- committee report}

%\clearpage
\section
{Introduction}

Preprints provide an opportunity for astronomers to share their results 
quickly with the astronomical community, in the six to nine month period 
between acceptance of a paper for publication and appearance of the paper 
in the journal.  Preprint distribution schemes vary widely, from sending 
preprints only upon request, to sending preprints to all astronomical
libraries plus a list of 50 or 100 astronomers who might remotely have
some interest in the subject of one's paper.  The preprint is a publishing 
phenomenon which is likely to persist as long as there is a many month 
delay between acceptance and publication of a journal paper, although in 
future some or all preprints may be distributed electronically rather than 
on paper.

A subcommittee of the AAS Publications Board, chaired by myself, was charged 
with examining the format of the AAS/WGAS Latex Macros package which
produces preprints.  I prepared a draft document which was reviewed by the
preprint subcommittee at the June 1992 AAS meeting.  This revised version 
incorporates the suggestions made then.

At the May 1992 Publications Board meeting, it was felt that the current 
preprint format generated by the AAS/WGAS Latex Macros, using the aaspp.sty 
style file, was unacceptable and that changes were desired.  Certain 
essential characteristics of a preprint were agreed upon.  First, the 
preprint should not mimic a page of one particular astronomical journal, 
such as the Astrophysical Journal.  This slights the other journals such 
as the Astronomical Journal and the Publications of the Astronomical 
Society of the Pacific, and furthermore blurs the distinction between the 
preprint and the final published version of the paper in the journal.
Second, the preprint should be easily read.  Other features also seem 
desirable, such as compactness to save trees and mailing charges, the 
ability to include figures and figure captions gracefully, and some 
flexibility so that more than one style of preprint could be accommodated 
for those with very divergent tastes.  Additionally, there is a clear need 
for something which will estimate how many pages a given text file will 
produce in the Astrophysical Journal Letters, due to the stringent page 
limit of that journal.  This estimate of the number of ApJ Letters pages 
might be possible to include as part of the output of a preprint formatter, 
although other approaches may be more desirable, such as a separate page 
length estimator program.

\section
{Observations}

I employed two techniques for surveying preferred preprint formats.  First, 
I examined 85 preprints I received in the last two years, to determine what 
the average preprint already looked like.  Since most astronomers have had 
access to a variety of word processors for many years, and thus have had 
the opportunity to select a format which suits their needs and tastes, it 
seemed this was a good indication of what most astronomers would prefer in
a preprint format.  This sample mostly (but not exclusively) contained 
preprints submitted to refereed journals, since preprints of papers intended
for conference proceedings were often in camera-ready format, reflecting not 
the tastes of the authors but of the publishers of the conference proceedings.
Second, I prepared several examples of possible preprint formats, using 
variations on the AAS/WGAS Latex Macros, and showed the output to various 
persons in my department to solicit opinions. 

\section
{Results}

\subsection{Survey of 85 Recent Preprints}

The overwhelming majority of preprints in the 1990--1992 sample surveyed 
were formatted with single columns of text per page (98 \%) and one page 
of text per preprint page (96 \%).  Obviously the AAS/WGAS Latex macros 
should respect this strong preference.

Most preprints had the figures at the end (87 \%) rather than interspersed 
in the text.  This may be partly due to preference and partly due to the 
current difficulty of including the figures within the text.

Astronomers were divided on whether to single space (42 \%) or double space
(58 \%) their text of their preprints.  Since there was no clear preference 
for either style, the AAS/WGAS Latex macros should make it extremely simple 
to choose either single spaced or double spaced output.

There was a wide divergence of opinions on how to divide up the title, 
abstract, and main text beginning with the introduction into pages.  Most 
preprints (62 \%) put title, abstract, and introduction on three separate 
pages, but a significant subset (14 \%) of preprints ran all three together 
with no page breaks.  Most astronomers (81 \%) preferred a separate title 
page whether or not they ran the abstract and introduction together.  Many 
preprints had two title pages: one printed on the institutional preprint 
cover or designed so that the title and authors were visible through a hole 
in the institution's preprint cover, and a second title page following in a 
format more suited to the author's tastes or needs.  A common feature of 
title pages was some indication of where the paper was to be published 
(title of the conference proceedings or name of the journal), its status 
(submitted, accepted, in press), and a date of some sort (date of the 
conference, date of submission to the journal, scheduled date for journal 
publication, or date of preprint mailing).  Clearly there should be a lot 
of flexibility in how the AAS/WGAS Latex macros format the title page.
Fortunately page breaks are simple to add or subtract in Latex.

\subsection{Survey of Seven OSU Astronomers}

The results from a survey of seven Ohio State astronomers, chosen because 
they happened to be in their offices when I walked down the hall, confirms 
the results of a larger preprint sample.  No-one liked a double column format.
There was a mix of opinions on whether or not to single space the text.
Most preferred the title, abstract, and main text on separate pages, but a 
significant minority felt strongly that the title and abstract should be 
on the same page.

\section
{Discussion}

The main feature that became apparent in this survey of preprint styles 
was that most authors were very lazy: they most often sent out preprints 
in the identical format that was used to submit the paper to the journal 
($\sim$ 60 \%), and if they made a change at all, it was to single space 
the text and perhaps remove some page breaks to conserve paper.  Thus, if 
the only change an author need make to produce a nice preprint is to include 
aaspp.sty instead of aasms.sty at the top of their paper, they will do it; 
but anything much more complex is likely not to be used.

\subsection{Text Format}

A preprint format which to first order satisfies all needs can be created 
with the current aasms.sty style file, using the $\backslash$tightenlines 
option (which makes the output single spaced instead of double spaced).
I propose the preprint format have as its default single-spaced lines and 
no page breaks between title, abstract, and main text, since it is simple 
for the author to insert $\backslash$clearpage wherever desired.  The main 
differences between the manuscript format and the preprint format, then, 
would be the spacing default (this should include tables; tables must be 
double spaced for the manuscript and should be single spaced as a default 
for the preprint), and the treatment of figures.  There should also be a 
simple Latex command to toggle the preprint format anywhere in the paper
between single spaced and double spaced, in case the author prefers to use 
one spacing for text and another for tables.

Most authors like to include extra information in their preprints along the 
lines of ``Invited paper presented at IAU Symposium 153, Galactic Bulges,
Ghent, Belgium, August 1992'' or ``Accepted for publication in AJ, November 
1992 issue.''  The preprint subcommittee prefers placing this information 
in the upper righthand corner of the title page, and suggests that a macro 
be created to do this.

\subsection
{Figures}

There are two possibilities for handling figures in preprints.  One is the 
traditional method of attaching a page or more of figure captions at the 
end of the paper, followed by full page figures.  This strategy has the 
advantage of providing large versions of figures which are later reduced 
to tiny sizes in the journal, and is popular among authors since it is 
also simple.  The other is to include the figures in the body of the text.
The AAS/WGAS Latex macros for preprints provide an opportunity to experiment 
with this latter technique.  The preprint subcommittee feels that placing 
the figures at the end of the text should be the default, but that the 
capability to move figures into the body of the text be included in the
AAS/WGAS Latex macros for preprints.

I suggest the AAS/WGAS Latex macros for preprints should place figures
in the text with one Latex command to include the contents of a Postscript 
file containing the figure, and a second Latex command to place the figure 
caption for each figure immediately under the figure, followed by white 
space separating the figure caption from the main text.  In some cases 
astronomers will not have the ability to create Postscript output for their 
figures, in which case a third Latex command is needed to leave the 
appropriate space (in cm or inches) for the figure above the figure caption,
in case authors desire to paste in figures by hand.  The preprint subcommittee 
suggests that the command to include a figure embedded within the text act 
as a comment to the journal suggesting where the figure should go in the 
paper, when used with aasms.sty, but this same command when used with 
aaspp.sty should actually incorporate the Postscript format figure into 
the text.

\section
{Summary}

The results of two surveys of astronomers concerning preprint formats are 
presented.  The consensus is that the default format should be single column
text, with the text either double spaced or single spaced as the author 
prefers.  The placing of page breaks between title, abstract, and main text 
should be at the author's discretion.  A macro is needed to place the
date and publication status of the preprint in the upper right hand corner
of the title page.  It may be useful for the preprint formatter to output 
an estimate of the number of pages of ApJ Letters format text that the 
preprint will produce.  While placing the figures at the end of the preprint
should be the default, the preprint format also provides an opportunity for
authors to experiment with including figures in the text.  This might 
be done with a command which in manuscript style acts as a comment to
the journal on figure placement, but which in preprint style acts to include 
a Postscript file containing the figure in the output.  

\acknowledgments
This wonderful document was prepared at the instigation of Caty Pilachowski.

\begin{references}

\reference{Allen}
Allen, N., Boisson, P., \& Chang, K.-Y.
1990, ApJ, 356, 907

\reference{Djor}
Djorestky, J., Estevez, M., \& Fujikama, K.
1991, AJ, 93, 473

\reference{Garcia}
Garcia, I., Heitzman, T., \& Innis, R.
1992, PASP, 102, 558

\reference{Jones}
Jones, E., Kaminski, C., \& Lee, H.
1990, AA, 127, 290

\reference{Monet}
Monet, C., Ng, T., \& Ortiz, A.
1991, MNRAS, 295, 684

\end{references}

\end{document}
