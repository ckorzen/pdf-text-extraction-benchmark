\documentstyle[agupp]{article}
   %\documentstyle[12pt,agums]{article}
   %\documentstyle[agums]{article}
   %\documentstyle[jgrga]{article}

   % AGUGUIDE.TEX -- User Guide for AGUTeX Package.
   %\author{Chris Biemesderfer}
   %\affil{Ferberts Associates}
   %\authoraddr{C. Biemesderfer, Ferberts Associates, P.O. Box 1180, 
   %   Oracle, AZ 85623}
   %\author{Becky Hunter}
   %\affil{American Geophysical Union}
   %\authoraddr{B. Hunter, 2000 Florida Avenue, NW, Washington, DC 20009}

\setcounter{secnumdepth}{4}
\hyphenation{com-pu-scripts}

\begin{document}
\twocolumn
\title{AGU\TeX\ Style Files for Manuscript Preparation}
\vspace{.11in}

The Publications Directorate of the American Geophysical Union 
has developed an author markup package.  This package will assist 
authors in preparing manuscripts that are intended for submission 
to AGU-affiliated journals.  It is important that the markup used 
by authors in electronic manuscripts be consistent and standardized 
so that such manuscripts can be widely accepted by the journals and 
so that they can eventually become part of the normal production 
procedures.

This guide contains basic instructions for creating manuscripts 
using the AGU\TeX\ package, which contains substyles to the 
standard \LaTeX\ {\tt article} style.  Authors are expected to 
be familiar with the editorial requirements of the journals so 
that they can make appropriate submissions, as well as to have 
at least a rudimentary knowledge of \LaTeX\ (for instance, knowing 
how to set up equations using \LaTeX\ commands).  It is unrealistic 
to provide a tutorial on \LaTeX; readers who are unfamiliar with 
the program are advised that they will need additional sources of 
information.  A number of useful publications are listed in 
the reference section of this guide.

\section{Introduction}

The AGU\TeX\ package was developed to provide AGU authors 
with a consistent means of preparing articles for submission 
to AGU journals.  The most important aspect of the package 
is that it defines the set of commands (called markup) that 
can be used to identify the structural elements of papers.  
AGU is not yet accepting electronic manuscripts, although 
this package is intended to be a step in that direction.  

The AGU\TeX\ package contains a number of \LaTeX\ style files 
that pro\-duce variously formatted pages.  There is a ``manuscript'' 
style, a ``preprint'' style, and four ``galley'' styles.  The 
manuscript style can be used to prepare papers for submission to 
AGU (i.e., for peer review and copyediting).  The preprint style 
produces a compact, two-column format suitable for distribution 
among colleagues.  The galley styles are used by authors who wish
to submit camera-ready galley pages to AGU for publication.

\section{Style Options}

The AGU\TeX\ project provides several format options using \LaTeX\ 
style files.  The primary motivation behind this idea is to allow 
sufficient flexibility for individuals to (1) prepare manuscripts
for submission, (2) distribute ``pretty'' preprints, and (3) generate 
camera-ready galleys with this package.

Editors prefer a large typeface with space between typed lines 
for technical editing marks and wide margins for editor and author 
comments.  The use of the {\tt agums} substyle will produce 
double-spaced, full-width output by default.  

Basic preprints can be created by including the \verb"\tighten" 
command in the document preamble when the {\tt agums} substyle is 
used; its effect is to reduce the vertical spacing in the text. The 
rest of the formatting will be identical to the manuscript style.

It may be preferable for preprints to be set in two columns; the 
{\tt agupp} substyle produces two-column pages.  This style can 
be used as is, but it can also serve as a point of departure for
\LaTeX\ style writers at institutions that want preprints of this 
general nature.

Finally, there are style files for preparing camera-ready ``galley''
pages with the AGU\TeX\ package.  A galley sheet contains printed
material typeset for a journal, but only one column is printed 
on a page.  Complete pages are pasted up at AGU.  

The substyle [{\tt jgrga}] creates galley styles for three journals:  
the {\it Journal of Geophysical Research}, {\it Global Biogeochemical 
Cycles}, and {\it Paleoceanography}.  Use substyle [{\tt grlga}] for 
{\it Geophysical Research Letters}, [{\tt tecga}] for {\it Tectonics}, 
and [{\tt radga}] for {\it Radio Science} manuscripts.

\section{General Preparation Instructions}

Computer manuscripts must include all the necessary components,
for example, a title, the author names and their affiliations,
an abstract, and a main body, in the proper order according to 
the editorial requirements of the journal.  It is the author's 
responsibility to ensure that the article conforms to all editorial 
specifications regarding content and organization, mathematical 
formulae, chemical names, etc.

In the sections that follow, we review some essential 
procedures to be used when preparing \TeX\ input.

\subsection{Running Text}

Printing is different from typewriting, and \TeX\ is different
from other word processing tools.  This section consists of
reminders (admonitions) about things that require special
attention so that \TeX\ can properly format the input.

The ends of words and sentences are marked by white space.  
It does not matter how many spaces are typed; one is as
good as 100.  \TeX\ treats the end of a line in the input 
file as a space.

Paragraphs are separated by blank lines.  Do not hyphenate 
words in the input file; \TeX\ takes care of hyphenation 
automatically.  Continue to hyphenate modifiers within a 
line of text, for example, ``author-prepared copy.''

Quotation marks should be typed as pairs of opening and
closing single quotes, for example, {\tt ``quoted text''};
do not use double quotes ({\tt "bad form"}).

Do not underline.  Slanted text is never used, and italics 
should only be used for math variables and text citations.

A number of common characters are interpreted as commands,
and these must be entered specially, by preceding them with
a backslash (\verb"\"): \$ \& \% \# \{ and \} must be typed
\verb"\$" \verb"\&" \verb"\%" \verb"\#" \verb"\{" and \verb"\}".

Authors should refrain from adding vertical or horizontal space.
Concentrate on the content of the document and identification of 
its components with the structural markup commands.  Authors should  
avoid creating new commands by using \LaTeX\ commands that create 
``private'' markup commands.

\subsection{Math}

Mathematical expressions that are part of the running text are
delimited by single dollar signs (\$), for example, \verb"$\pi r^2$" 
yields $\pi r^2$.  To get the appropriately sized superscript or 
subscript in the Roman font, use the \verb"\rm" command, for example, 
\verb"$J_{\rm HF}(t)$" produces $J_{\rm HF}(t)$.

Displayed equations can be delimited in several ways.  The most 
concise markup is to bracket the equation between \verb"\[" and 
\verb"\]" commands, which is equivalent to placing the formula 
in a {\tt displaymath} environment.  These markup commands will 
produce unnumbered equations.

Numbered equations can be typeset by typing the formula in
an {\tt equation} environment.  A series of related equations 
that need vertical alignment, for example, a derivation where 
alignment is wanted on the equals sign (=), can be typeset in 
an {\tt eqnarray} environment.

While it is possible for authors to assign their own equation 
numbers, it is easier to let \LaTeX\ number them automatically.
By default, \LaTeX\ will number equations sequentially from the
beginning to the end of the paper.

\subsection{Cross-referencing}

Cross-referencing equations, tables, and figures in text depends 
on the use of ``tags,'' which are defined by the user.  The 
\verb"\label" command is used to define cross-reference tags 
for \LaTeX; \verb"\ref" is used to refer to them.  Tags are 
simply text strings that serve to label equations, tables, and 
figures so that they may be referred to symbolically in the 
text.  Authors should place \verb"\label" commands immediately 
after the markup command that starts the structure being 
referenced.  References to page numbers should not be made.

\LaTeX\ keeps track of autonumbered counters and cross-reference
information by maintaining an auxiliary file in the same working
directory as the source file.  The auxiliary file will have an
extension of {\tt .aux}.  This file should not be deleted, since
subsequent \LaTeX\ processing uses the auxiliary data to number 
items and to resolve references.

The auxiliary file mechanism makes it necessary to run \LaTeX\ on 
a given source file more than once to ensure that the cross-reference 
information has been resolved properly.  This is most evident when 
changes are made that affect the number or the placement of equations, 
tables, or figure captions.  \LaTeX\ will typically issue a warning 
message that advises the user to ``rerun to get cross-references 
right,'' in which case, one should \LaTeX\ the file again.  If the
error message appears after two successive \LaTeX\ runs, it is likely 
that a reference has been made to an undefined label.

\section{Command Descriptions}

This section describes commands in the AGU\TeX\ markup package 
that an author might enter in an electronic manuscript.  In the 
interest of completeness, all structural markup that is needed to 
identify components is discussed.  The commands will be described 
in roughly the same order as they would appear in a manuscript.  
The reader will probably find it helpful to examine the sample 
paper (\verb"sample.tex") as well. 

\subsection{Preamble}

Most documents processed with a formatter have a collection of commands 
at the beginning of the file that establish settings for global parameters 
and formatting; this initialization section is sometimes called the preamble.  
In \LaTeX\ manuscripts the preamble is that portion of the file before the 
\verb"\begin{document}" command.
\clearpage
The first piece of markup in the manuscript must 
declare the overall style of the document:
\begin{center}
\begin{quote}
\verb"\documentstyle[12pt,agums]{article}"
\end{quote}
\end{center}
The \verb"\documentstyle" command must appear first in any \LaTeX\ 
file.  The command shown above specifies the main style to be the {\tt 
article} style using 12-point fonts and the {\tt agums} substyle.  
The manuscript style uses a 12-point font to prevent illegibility 
due to exceedingly long lines (too many characters on a line make 
it hard to read).  If the font size is smaller than 12 points, the 
{\tt agums} substyle issues a warning message and will automatically 
reset the size to 12 points; the file will still be processed.

Authors may include a \verb"\tighten" declaration in the preamble to 
produce a somewhat denser manuscript.  Tightening the vertical spacing 
in the text results in output that may be attractive for distribution 
to colleagues who are primarily interested in reading the paper, as 
opposed to copyediting it.  If authors prefer not to insert this markup 
directly, the same effect can be achieved by using the \verb"tighten" 
command in the document substyle.
\begin{center}
\begin{quote}
\verb"\documentstyle[agums,tighten]{article}"
\end{quote}
\end{center}
Authors may wish to adjust vertical spacing within a preprint;
for instance, double-spacing text while single-spacing tables.
Authors who want to alternate between single and double
spacing in the manuscript may use the following commands.
\begin{quote}
\verb"\singlespace"\\
\verb"\doublespace"
\end{quote}
These commands differ from the \verb"\tighten" command, which 
is intended to be used in the preamble of the manuscript.  

Double-spaced output for referees and copy editors is the main 
objective of the {\tt agums} style, hence the double-spacing occurs by 
default.  Neither the \verb"\tighten" command nor the \verb"tighten"
substyle should be used for manuscripts submitted to the editorial 
office for scientific review.

\subsubsection{Two-column format.}
The {\tt agupp} substyle has the principle function of setting 
up two-column output (for distribution to colleagues).
\begin{quote}
\verb"\documentstyle[agupp]{article}"
\end{quote}
Although it is quite obvious, it is important to remember that 
text lines are considerably shorter when two of them are typeset 
side by side on a page.  Long equations, wide tables and figures, 
etc., may not typeset in this format without some adjustments.  The 
expenditure of great effort to adapt copy and markup for two-column 
pages is probably counterproductive.  Remember that the main goal of 
the AGU\TeX\ package at this point is to produce ``correct'' draft 
(or referee) format pages; it is the responsibility of the editors 
and publishers to produce publication format papers for the journals.

The {\tt agupp} substyle does not impose a format for the 
article's front matter, although there is often merit in 
setting the title, author, abstract, and keyword material 
on a separate page at full text width.  The author may 
place the \verb"\twocolumn" command wherever desired.
\begin{quote}
\verb"\twocolumn"
\end{quote}
Please note that the two-column format begins at the point 
the \verb"\twocolumn" command appears in the text.  If that 
point occurs before the front matter then the title, author, 
etc. will be typeset in two-column mode along with the rest 
of the paper; that's how this manual is prepared, for example.
To produce ``pretty'' output it is probably desirable to put 
the \verb"\twocolumn" command after the abstract and keywords, 
just before the body of the paper. If \verb"\twocolumn" is not 
specified explicitly in {\tt agupp} style documents, the 
introductory material of the paper will be set in one-column 
mode; the two-column mode will engage with the first 
\verb"\section" command.

\subsubsection{Galley styles.}
After an article has been accepted for publication the author may 
decide to provide a camera-ready manuscript for publication.  The 
AGU\TeX\ package contains style files that produce such galleys for 
the following AGU journals:
\begin{center}
\begin{tabular}{ll}
\tt grlga & GRL\\
\tt jgrga & JGR, GBC, and {\it Paleoceanography} \\
\tt radga & {\it Radio Science} \\
\tt tecga & {\it Tectonics} \\
\end{tabular}
\end{center}
These are all substyles to the standard \LaTeX\ {\tt article} 
style and should be selected as a substyle option.  For instance, 
an author who wished to prepare galley pages for JGR would 
specify the style of the document in the following manner:
\begin{quote}
\verb"\documentstyle[jgrga]{article}"
\end{quote}

\subsubsection{Slug line data.}
Journal and article identification information is established by the 
editorial staff.  The following markup will be used by personnel at 
the editorial office to record slug line data.  These markup commands 
should appear in the manuscript preamble and can be used to facilitate 
communication between editorial office, author, and publisher.  For 
preprints and manuscripts in draft/referee format the slug line
information is irrelevant, and the data usually are not formatted 
or printed.
\begin{quote}
\verb"\received{RECEIPT DATE}"\\
\verb"\revised{REVISION DATE}"\\
\verb"\accepted{ACCEPT DATE}"
\end{quote}
Receipt and acceptance dates (or blank lines representing them) are 
printed on {\tt agums} articles, but the {\tt agupp} style does not 
print either blank lines or dates.  Authors generally do not know 
what these dates are, and there is no need for them to include the 
\verb"\received" and \verb"\accepted" commands in non galley 
manuscripts.  For galley styles the necessary information will be 
transmitted to the author prior to the formatting of the final copies.  
The information contained in these commands will appear at the end of 
the article after the references and author's addresses.  

\begin{quote}
\verb"\forcesluginfo"
\end{quote}

If there is no reference list (e.g., in an editorial), the author 
must manually force the slug information onto the last page with the 
\verb"\forcesluginfo" command.  This command should be used only when 
there is no reference section, and it should appear in the file after 
the body of the paper at the point where the reference section would 
normally print.

\begin{quote}
\verb"\journalid{VOL}{JOURNAL DATE}"\\
\verb"\articleid{START PAGE}{END PAGE}"\\
\verb"\paperid{MANUSCRIPT ID}"
\end{quote}

The \verb"\journalid" and \verb"\articleid" commands are used 
to identify the volume and page numbers of a scheduled article.  
The manuscript identification number (used to track the manuscript)
is specified in the \verb"\paperid" command.  For AGU journals this 
identification is two digits indicating the year the manuscript is
received at AGU, two letters indicating the journal, and five more 
digits indicating in what order the manuscript was received by the
Publications Department.  Thus, manuscript 94JA00001 would be the 
first manuscript received for {\it JGR-Space Physics} in 1994.

\begin{quote}
\verb"\cpright{TYPE}{YEAR}"\\
\verb"\ccc{CODE}"
\end{quote}

Copyright information should be specified with the \verb"\cpright" 
command.  The ``type'' of copyright and the corresponding year are 
given in \verb"\cpright"; valid copyright types are ``AGU,'' 
``Crown,'' and ``PD:''
\begin{center}
\begin{quote}
\begin{tabular}{l@{\quad}p{3in}}
\tt AGU & Copyright is assigned to AGU.\\
\tt Crown & Copyright is re\-served by a Crown \\
& (British Commonwealth) government.\\
\tt PD & The article is in the public domain.\\
\end{tabular}
\end{quote}
\end{center}
The copyright type is case-sensitive, so the type string must be
entered exactly as shown above.  The Copyright Clearance Center 
code may be given in the \verb"\ccc" command.  The code is viewed
as regular text, so any special characters, notably ``\$,'' must 
be properly specified.

Authors are asked to supply running head information.  There are 
two different kinds of data in the running heads.  The left head 
contains an author list (last names, with more than three authors 
truncated as ``et al.''), while the right head is an abbreviated 
form of the paper title.
\begin{quote}
\verb"\lefthead{TEXT}"\\
\verb"\righthead{TEXT}"
\end{quote}
These items should be kept as short as possible; the abbreviated 
title should be no more than 65 characters, including spaces.

Authors who wish to include a short remark on the title page,
such as the name and date of the journal in which an article
has been scheduled, may do so with the following command.
\begin{quote}
\verb"\slugcomment{TEXT}"
\end{quote}
In the {\tt agums} style, such comments appear after the dates on 
the title page; in the {\tt agupp} style, they are placed in the 
upper left corner of the title page.  These comments do not appear 
in the camera-ready styles.

\subsection{Starting the Main Body}

The preamble is only a control section, so the markup that appears 
in the preamble is actually typeset elsewhere in the manuscript.  
Authors must include a \verb"\begin{document}" command to identify 
the beginning of the main portion of the manuscript.

\subsection{Title, Byline, Abstract, etc.}

Title and author are identified by the markup commands \verb"\title" 
and \verb"\author".  The authors' principal affiliation is specified 
with a separate command \verb"\affil".  Each \verb"\author" command 
should be followed by a corresponding \verb"\affil". 
\begin{quote}
\verb"\title{LUCID TEXT}"\\
\verb"\author{NAME(S)}"\\
\verb"\affil{AFFILIATION}"\\
\verb"\authoraddr{ADDRESS}"
\end{quote}

Line breaks may be specified in the title with the double backslash 
(\verb"\\") command.  Long titles will be broken automatically, so 
the \verb"\\" markup is not required.  If the title is explicitly 
broken over several lines, AGU journals prefer the ``inverted pyramid'' 
style.  In this style the longest line is the first (or top) line, and 
each succeeding line is shorter.  The text of the title should be entered 
in mixed case.  Footnotes are not permissible in titles.

Authors' names are given in \verb"\author" commands and should be 
entered in mixed case.  Authors who have the same primary affiliation 
should be specified in a single \verb"\author" command.  Each author 
group (\verb"\author" command) should be followed by an \verb"\affil" 
command, giving the principal affiliation of that author.  The 
affiliation, which does not include street numbers or zip codes, 
can be broken over several lines by using the \verb"\\" command 
to indicate the line breaks. 

If there are more than three author and affiliation groupings then all 
of the authors' names must be specified in a single \verb"\author" command.  
The affiliations should then be specified with the \verb"\altaffilmark" 
command described below.  In this case, no \verb"\affil" commands would be 
used, and the affiliations would all be listed in a footnote block.  The 
style file performs this formatting.  There is an example of this format 
in the \verb"sample.tex" document.

Authors often have affiliations in addition to their principal 
employer.  These alternate affiliations are specified with 
\verb"\altaffilmark" and \verb"\altaffiltext" commands.  These 
act like \LaTeX's \verb"\footnotemark" and \verb"\footnotetext" 
commands.  The \verb"\altaffilmark" command is appended to the 
author name in the \verb"\author" lists and generates superscript 
identification numbers.  The text for the individual alternate 
affiliations is generated by the \verb"\altaffiltext" command.
\begin{quote}
\verb"\altaffilmark{TAG NUMBER(S)}"\\
\verb"\altaffiltext{NUMERICAL TAG}{TEXT}"
\end{quote}
Authors should make sure that the \verb"\altaffilmark" 
numbers attached to authors' names correspond to the 
correct alternate affiliation, i.e., that each tag number 
matches the numerical tag for the corresponding text.

Postal addresses for individual authors may be specified in 
\verb"\authoraddress" commands.  This command does not produce 
any formatted text in the {\tt agums} and {\tt agupp} styles.  
In the galley styles the address information is formatted and 
appears at the end of the article after the ref\-erences.  This 
information should be placed immediately after the copyright 
information in the preamble of the article.

\subsubsection{Abstract.}The manuscript abstract should be enclosed 
in an {\tt abstract} environment.
\begin{quote}
\verb"\begin{abstract}"\\
Assorted abstract text.\\
\verb"\end{abstract}"
\end{quote}

\subsection{Sections}

The {\tt article} style for AGU manuscripts 
supports three levels of sectioning.
\begin{quote}
\verb"\section{HEADING}"\\[.5ex]
\verb"\subsection{HEADING}"\\[.5ex]
\verb"\subsubsection{HEADING}"
\end{quote}
Level one and level two heads should be given in caps and lower case 
with no periods.  Level three heads should capitalize the first word, 
end with a period, and should run into the text of the following 
paragraph.  Note that these commands delimit sections by marking the 
beginning of each section; there are no separate commands to identify 
the ends.  If it is imperative for sections to be numbered, the numbers
should be manually placed in the section headers in the format:  
\verb"\section{1. Introduction}".  \LaTeX's automatic section numbering 
should not be turned on since it does not produce AGU format.

If authors wish to include acknowledgments the AGU\TeX\ styles support an 
\verb"\acknowledgments" section.  This is generally the concluding section 
of the main body text.

\subsubsection{Appendices.}Appendices may be included with an 
\verb"\appendix" command.  Note that this command has no arguments 
and no ``end appendix'' command.  Sections in the appendix are marked 
with \verb"\section" commands containing the section headings, as 
described above.  The \verb"\appendix" command takes care of a number 
of internal housekeeping concerns, such as identifying sections with 
letters, resetting the equation counter, etc.

It is important to note that the \verb"\appendix" command causes all 
ensuing equation and table headers to use lettered captions.  This is 
to distinguish appendix tables and equations from the tables and equations 
that will appear in the main body of the manuscript.  Since AGU submission 
style asks for all tables to be printed at the end of a manuscript, 
manuscripts that contain both appendix sections and regular tables must 
use the \verb"\tablenum{}" command to number the regular tables.  This will 
not pose a problem for manuscripts with either no tables or no appendix 
section.  The \verb"sample.tex" manuscript includes examples of tables 
with \verb"\tablenum{}" commands.

\subsection{Citations}

Citations in the text may be called out with either a 
\verb"\markcite" command or a \verb"\cite" command.
\begin{quote}
\verb"\markcite{TEXT}"\\[.5ex]
\verb"\cite{TAG}"
\end{quote}
Which of these is used depends on the form chosen for the reference 
list (described below).  The \verb"\markcite" command corresponds to 
the use of the {\tt references} environment, while \verb"\cite" would 
be used in conjunction with {\tt thebibliography} environment.

The conventional method used by authors to manage the citations 
and reference list in a paper is a manual one.  For these authors 
\verb"\markcite" and the {\tt references} environment are appropriate.
The text supplied in the \verb"\markcite" command is the text that
will be inserted in the body at that point; this text should include
any necessary punctuation.

A more automated way of creating the reference list is with \LaTeX's {\tt 
thebibliography} environment, which identifies references using unique tags 
chosen by the author.  If this mechanism is used, the {\small TAG} given in 
the \verb"\cite" command must correspond to a {\small TAG} given in a 
\verb"\bibitem" command in the {\tt thebibliography} environment.  Examples 
of both styles are shown in the {\tt sample.tex} document.

Citations may be called out without any markup since they are usually 
just running text.  However, the use of \verb"\markcite" is still 
encouraged, even for journals where no special formatting of citation 
call outs is required.  This will make the electronic texts more useful 
if they are ever searched with on-line browsers.

\subsection{Reference Sections}

As discussed in the preceding section there are two main methods 
for managing citations and references.  Many authors are comfortable 
with entering properly formatted citations directly in the body of an 
article and then organizing the reference list themselves.  These 
authors would use \verb"\markcite" for citation call outs in the text 
and employ the {\tt references} environment for the reference list.  
\begin{quote}
\verb"\begin{references}"\\
\verb"\reference" Assorted bibliographic data. \\
\verb"."\\
\verb"."\\
\verb"\end{references}"
\end{quote}
The {\tt references} environment sets off the list of references 
and adjusts spacing parameters.  Each reference is preceded by a 
\verb"\reference" command.  

Bibliographic data must conform to the standards of the journal.  It is 
the responsibility of the author to place the bibliographic information 
in the proper order with the correct punctuation; the information will 
be typeset as is, i.e., with no size or style changes.

If authors choose to mark citations with \verb"\cite" they should use
\LaTeX's {\tt thebibliography} environment and associate references 
with the \verb"\bibitem" command.  The \verb"\cite"--\verb"\bibitem" 
mechanism uses symbols to associate citations and references.  Authors 
must create the citation label for each reference in proper journal 
format within the \verb"\bibitem" command.
\begin{center}
\begin{quote}
\verb"\begin{thebibliography}"\\
\verb"  \bibitem[LABEL]{TAG}"\verb"\reference" Assorted bibliographic data. \\
\verb"   ."\\
\verb"   ."\\
\verb"\end{thebibliography}"
\end{quote}
\end{center}
The {\small LABEL} must adhere to journal standards, for example, [{\it 
Abt}, 1986].  It is not possible to use \verb"\bibitem"s within the {\tt 
references} environment, nor will the \verb"\cite" commands work properly 
in the main body if the \verb"\bibitem"s are not specified properly.  This 
technique can be tricky, and there are limitations on the way the citation 
{\small LABEL} is formatted.  Authors are advised to consult the \LaTeX\ 
manual [{\it Lamport}, 1985].

Citation management can be complex, and many systems have been developed 
to assist authors in preparing bibliographies.  The program that manages 
references within the \TeX\ family is called BIB\TeX, and it is designed 
to work in conjunction with the citation and reference list capabilities 
of \LaTeX.  At the present time there is no compelling reason to force an 
implementation based on BIB\TeX, although it should be possible to use 
BIB\TeX\ to build reference lists with if authors wish to do so.  (It is 
also possible to define a bibliographic style for BIB\TeX\ so that 
citations and reference lists are automatically formatted.)

A footnote stating that the article was prepared with the AGU\TeX\ 
package appears on the last page of references.

\subsection{Equations}

\LaTeX's standard displayed math environments allow authors to typeset 
displayed equations in many ways.  The following three are probably used 
most:
\begin{quote}
\verb"\begin{displaymath}"\\
\verb"\end{displaymath}"\\[.8ex]
\verb"\begin{equation}"\\
\verb"\end{equation}"\\[.8ex]
\verb"\begin{eqnarray}"\\
\verb"\end{eqnarray}"
\end{quote}
The {\tt displaymath} environment breaks out a single, unnumbered formula.  
The {\tt equation} environment creates a formula similar to {\tt displaymath}, 
but also invokes \LaTeX\'s automatic numbering.  The {\tt eqnarray} 
environment allows you to vertically align several formulae. 

\begin{quote}
\verb"\begin{mathletters}"
\end{quote}

Authors occasionally wish to group related equations together and identify 
them with letters appended to the same equation number, as opposed to each 
having a separate numeral.  Such related equations should still be set in 
{\tt equation} or {\tt eqnarray} environments (whichever is appropriate may 
be used), and this grouping should then be placed within a {\tt mathletters} 
environment.

\begin{quote}
\verb"\eqnum{TEXT}"
\end{quote}

It is possible to override \LaTeX's automatic numbering within 
{\tt equation} or {\tt eqnarray} environments.  When \verb"\eqnum" 
is specified inside an {\tt equation} environment or on a particular 
equation within an {\tt eqnarray}, the text supplied as an argument to 
\verb"\eqnum" is used as the equation identifier.  The \verb"\eqnum" 
command must be used inside the environment.  \LaTeX's equation counter 
is not incremented when \verb"\eqnum" is used.  

When unnumbered equations are desired, authors can use either 
the {\tt displaymath} environment (for single displayed equations) 
or the \verb"\nonumber" command, (which is placed before the 
equation delimiter [\verb"\\"]).  \LaTeX's equation counter is 
not incremented when \verb"\nonumber" is used.

If the use of \verb"\eqnum" or \verb"\nonumber" causes \LaTeX's 
equation counter to number equations in the wrong sequence, the 
counter may be reset to a particular value by using a 
\verb"\setcounter{equation}{NUMBER}" command.  The equation 
counter should be set to the number corresponding to the last 
equation that was formatted, so it is most appropriate for this 
command to appear immediately after an {\tt equation} or {\tt 
eqnarray} environment ends.  This command must be used outside 
the math environments.

\subsubsection{Abbreviations for journals.}  
Markup commands have been created for many of the often-referenced 
journals.  Authors may use these markup names as shorthand rather 
than having to look up a particular journal's specific abbreviation.
To make sure there will be no confusion the following list shows full 
journal names, but the codes will produce abbreviations appropriate 
for AGU bibliography style.  Thus \verb"\apj" would produce ``\apj''
\vspace{-10pt}
\begin{center}\begin{tabular}{ll}
\verb"\apj"    & {\it Astrophysical Journal}\\
\verb"\bams"   & {\it Bulletin of the American Meteorological}\\
	       & \hspace{2em}{\it Society}\\
\verb"\bssa"   & {\it Bulletin of the Seismological Society of}\\
               & \hspace{2em}{\it America}\\
\verb"\dsr"    & {\it Deep Sea Research}\\
\verb"\eos"    & {\it Eos (Transactions of the American}\\
               & \hspace{2em}{\it Geophysical Union)}\\
\verb"\epsl"   & {\it Earth and Planetary Sciences Letters}\\
\verb"\gca"    & {\it Geochimica and Cosmochimica Acta}\\
\verb"\gjras"  & {\it Geophysical Journal of the Royal}\\
               & \hspace{2em}{\it Astronomical Society}\\
\verb"\grl"    & {\it Geophysical Research Letters}\\
\verb"\gsab"   & {\it Bulletin of the Geological Society of}\\
               & \hspace{2em}{\it America}\\
\verb"\jatp"   & {\it Journal of Atmospheric and Terrestrial}\\
               & \hspace{2em}{\it Physics}\\
\verb"\jgr"    & {\it Journal of Geophysical Research}\\
\verb"\jpo"    & {\it Journal of Physical Oceanography}\\
\verb"\mnras"  & {\it Monthly Notices of the Royal}\\
               & \hspace{2em}{\it Astronomical Society}\\
\verb"\mwr"    & {\it Monthly Weather Review}\\
\verb"\pepi"   & {\it Physics of the Earth and Planetary}\\
               & \hspace{2em}{\it Interiors}\\
\verb"\pra"    & {\it Physical Review A: General Physics}\\
\verb"\prb"    & {\it Physical Review B: Solid State}\\
\verb"\prc"    & {\it Physical Review C: Nuclear Physics}\\
\verb"\prd"    & {\it Physical Review D: Particles and}\\
               & \hspace{2em}{\it Fields}\\
\verb"\prl"    & {\it Physical Review Letters}\\
\verb"\qjrms"  & {\it Quarterly Journal of the Royal}\\
               & \hspace{2em}{\it Meteorological Society}\\
\verb"\rg"     & {\it Reviews of Geophysics}\\
\verb"\rs"     & {\it Radio Science}\\
\verb"\usgsof" & {\it U.S. Geological Survey Open File}\\
               & \hspace{2em}{\it Report }\\
\verb"\usgspp" & {\it U.S. Geological Survey Professional}\\
               & \hspace{2em}{\it Papers}\\
\end{tabular}
\end{center}

\clearpage

\subsection{Tables}

AGU\TeX\ supports two table mechanisms: (1) \LaTeX's standard {\tt 
table} and {\tt tabular} environments and (2) a {\tt planotable} 
environment that facilitates the formatting of lengthy tabular 
material.  Short tables may be marked up using either mechanism; 
long tables will require the use of {\tt planotable}.

\LaTeX\ permits fairly complex tables with arbitrary spacing, straddle 
heads, rules, etc.  Authors who need to specify complicated column 
headings are advised to consult the \LaTeX\ manual [{\it Lamport}, 1985] 
for details.  Most of the capabilities are applicable to AGU\TeX's {\tt 
planotable} as well as \LaTeX's {\tt tabular} environment.

\begin{quote}
\verb"\tablenum{TEXT}"
\end{quote}

It is possible to override \LaTeX's automatic numbering within both the
{\tt table} and {\tt planotable} environments.  When a \verb"\tablenum" 
command is specified inside one of the {\tt table} environments, the 
{\small TEXT} supplied as an argument to \verb"\tablenum" is used as 
the table identifier.  \LaTeX's equation counter is not incremented 
when \verb"\tablenum" is used.  The \verb"\tablenum" command must be 
used inside the {\tt table} environments.

\subsubsection{Short tables.}
Short tables (smaller than one manuscript page) may be marked and 
composed using the standard \LaTeX\ tools for tables.  Tables should 
appear within the \verb"\begin{table}" and \verb"\end{table}" environment.

The {\tt table} environment encloses not only the tabular material but 
also any title (caption) or footnote information associated with the table.

Captions for short tables should be indicated with \verb"\caption{TEXT}" 
commands.  Tables will be identified with arabic numerals, for example, 
``Table 2.''  The identifying number and period are generated automatically 
by the \verb"\caption" command within the \verb"table" environment.

Tabular information is typeset within the {\tt tabular} environment.
\begin{quote}
\verb"\begin{tabular}{COLS}"\\
\verb"\end{tabular}"
\end{quote}
where {\small COLS} specifies the justification for each column.
One of the letters ``l'', ``c'', or ``r'' is given for each column,
indicating left, center, or right justification.  Consult the 
\LaTeX\ manual [{\it Lamport}, 1985] for details about using the 
{\tt tabular} environment to prepare tables.  There should be only 
one {\tt tabular} table per {\tt table} environment.  

Authors may use the \verb"\tableline" command within {\tt tabular} 
environments.  This command produces horizontal rule(s) after the title, 
after the column headings, and after the table data.  There should be 6 
points of space between each rule and each line of text, including any 
table notes or footnotes.  Vertical rules should not be used.

\subsubsection{Long tables, plano tables.}\label{tbl-sec}
This section describes the use of the {\tt planotable} environment.
This environment is so named because it was originally developed 
to aid authors in preparing camera-ready tables for an article; 
such camera-ready material would be produced in the same way 
as planographic figures, hence the term ``plano table.''

The {\tt planotable} environment has several capabilities, somewhat 
above and beyond \LaTeX's {\tt tabular} environment, that facilitate 
the formatting of tables.  Among these are breaking long tables across 
pages, choosing a table width, and specifying comments and references 
for tables.

The {\tt planotable} environment uses \LaTeX's familiar 
\verb"\begin" and \verb"\end" constructs.
\begin{quote}
\verb"\begin{planotable}{COLS}"\\
\verb"\end{planotable}"
\end{quote}
{\small COLS} specifies the justification for each column. 
One of the letters ``l,'' ``c,'' or ``r'' is given for each 
column, indicating left, center, or right justification.  

There are several items in a {\tt planotable} environment 
that must be given before the table data.
\begin{quote}
\verb"\tablewidth{DIMEN}"\\
\verb"\tablecaption{TEXT}"
\end{quote}
The width of a plano table is set by a dimension specified in the 
\verb"\tablewidth" command; the default width is the width of the 
body text.  The table can be set to its natural width by specifying 
a {\small DIMEN} of 0pt.  When preparing camera ready galleys, tables 
must be set to specific widths.  The permissible values are as follows:
\begin{center}
\begin{tabular}{p{0.5in}c}
\underline{Style} & \underline{Table Width Command}\\
\tt jgrga & 20pc, 30pc, 41pc, or 48--57pc\\
\tt grlga & 20pc, 30pc, 41pc, or 48--57pc\\
\tt tecga & 20pc, 30pc, 41pc, or 48--57pc\\
\tt radga & 19pc, 30pc, 37pc, or 43--51pc\\
\end{tabular}
\end{center}

The caption (title) of a plano table is specified in the 
\verb"\tablecaption" command.  The intent is for the text 
of the \verb"\tablecaption" to be brief; explanatory notes 
may be specified in the end notes to the table (see 
information on \verb"\tablecomments", below).

\begin{quote}
\verb"\tablehead{TEXT}"\\
\verb"\colhead{HEADING}"
\end{quote}

Each column heading is specified in a \verb"\colhead" command, 
which should appear within a \verb"\tablehead" command.  The 
\verb"\colhead" command centers the headings on the natural width 
of the column.  This is the typical disposition of column headings, 
and the use of \verb"\colhead" is encouraged.  Each data column 
should have a heading.  If more complicated column headings are 
required, any valid {\tt tabular} commands that constitute a proper 
head line for the table may be used.  The double backslash command 
(\verb"\\") may be used to end table headings, and extra vertical 
spacing can be inserted with bracket commands (\verb"\\[.5ex]").

It is possible for a complicated table heading to overflow 
the vertical space allotted for the table heading.  The fraction 
of the page allocated for the table heading may be changed with 
\verb"\tableheadfrac".  The {\small NUM} argument to 
\verb"\tableheadfrac" should be the decimal fraction of the page 
used for heading information.  The default value is 0.1, meaning 
that 10\% of the page height is reserved for the table heading.  
It should rarely be necessary to change this value.

\begin{quote}
\verb"\startdata"
\end{quote}

After the table title and column headings are specified, data lines
can be entered.  The beginning of the data lines is indicated by a 
\verb"\startdata" command.  This command formats the column headings, 
engages the tabular formatting, and produces a table title (caption). 

\begin{quote}
\verb"\nl"\\
\verb"\vspace{DIMEN}"
\end{quote}

Data elements within a row of the table are separated with {\tt 
\&} (ampersand) characters.  The end of each row is indicated with 
a \verb"\nl" command.  Extra vertical space can be inserted between 
rows with a \verb"\vspace" command; the argument is a dimension 
and may be specified in any units that are legitimate in \LaTeX.  

\begin{quote}
\verb"\tablebreak"\\
\verb"\nodata"
\end{quote}

If a page break needs to be forced in a plano table, \verb"\tablebreak" 
should be used instead of \verb"\nl".  This is sometimes necessary when 
several rows of data are associated with a single object or item; such 
logical groupings should not be broken across pages, and in these cases 
\verb"\tablebreak" can be used to ensure that the page breaks are rational. 

Table elements for which there are no data may contain a \verb"\nodata" 
command; an appropriate symbol will be placed in that data element.

A ``specialty'' head is provided within the plano table body.  The 
\verb"\cutinhead{TEXT}" command creates a centered, italic heading 
that is centered on the table width.  All of these formatting 
particulars are managed by the style files; the author need 
only type the appropriate text.

\subsubsection{Table footnotes.}
AGU\TeX\ supports footnotes (endnotes) that are associated with tables; this 
support applies to both the {\tt planotable} environment and the standard 
\LaTeX\ {\tt table} environment.  Footnotes for tables are usually identified 
by lowercase letters rather than numbers.  Marking and assigning associated 
text is achieved with the \verb"\tablenotemark" and \verb"\tablenotetext" 
commands, in which the note identifier is required (cf.\ \verb"\altaffilmark" 
and \verb"\altaffiltext").  Output from \verb"\tablenotetext" is triggered by 
\verb"\end{table}", so be sure the \verb"\end{table}" command appears last.

\begin{quote}
\verb"\tablenotemark{TAG}"\\
\verb"\tablenotetext{TAG}{TEXT}"
\end{quote}
Note that the {\small TAG} used for the mark should be the 
same as the {\small TAG} for the corresponding text.  It is 
the responsibility of the author to make sure they correspond.

\begin{quote}
\verb"\tablecomments{TEXT}"
\end{quote}

Occasionally, authors wish to append a short paragraph of explanatory 
notes that pertain to the entire table, and sometimes authors tabulate 
items which have corresponding references, and it may be desirable to 
associate these references with the table rather than (or in addition 
to) the formal reference list.  Both of these types of table notes should
be placed in a \verb"\tablecomments" command.  Only one paragraph of 
material is permitted at the end of a table, (excluding called-out 
references).  If both references and notes are desired they should 
be placed together in the \verb"\tablecomments" command.  See {\tt
sample.tex} for an example of this command.

The table endnotes are coupled to the table in which they occur 
rather than being associated with a particular page, and they are 
printed with the table (relatively close to the caption) instead of 
appearing at the extreme bottom of the page.  This is done to ensure 
that the notes wind up on the same page as the table, since tables 
are floats and can migrate from one page to another.

\subsection{Figures}

Do not integrate figures into text.  Figure captions should be produced 
by using the \verb"\caption" command within an otherwise empty {\tt figure} 
environment.  
\begin{quote}
\verb"\begin{figure}"\\	
\verb"\caption{TEXT}"\\
\verb"\end{figure}"
\end{quote}
When the {\tt figure} environment is used, the figure identification, for 
example, ``Figure 1,'' is generated automatically by the \verb"\caption" 
command.  The captions should be placed after the reference section.  It 
is acceptable for several widely spaced figure captions to appear on the
same page.

\begin{quote}
\verb"\figurewidth{DIMEN}"
\end{quote}

In galley styles the width of a figure caption is defined by the 
\verb"\figurewidth" command.  The default width is the width of the 
body text.

When preparing galleys, figure captions must be set to 
specific widths.  The permissible values are as follows:
\vspace{-1pc}
\begin{center}
\begin{tabular}{p{1.4in}c}
\underline{Style} & \underline{Figure Width Command}\\
\tt jgrga, grlga, tecga  & 20pc, 35pc, or 41pc\\
\tt radga  & 19pc, 33pc, or 38pc\\
\end{tabular}
\end{center}

It is possible to use a \verb"\figurenum{TEXT}" command to override 
\LaTeX's automatic numbering within the {\tt figure} environment.
When a \verb"\figurenum" command is specified inside a {\tt figure} 
environment, the {\small TEXT} is used as the figure identifier.  
\LaTeX's equation counter is not incremented when \verb"\figurenum" 
is used.

Footnotes are not supported for figures.

\subsection{Fonts}
Authors with postscript printers may use AGU\TeX\ style files to produce 
a Times Roman typeface.  Authors (or their system managers) must locate and 
install a dvips program.  Dvips is a freely redistributable PostScript driver 
for device independent files.  The program can be found on several different 
anonymous ftp sites.  Once the program is correctly installed, authors should 
include the word ``times'' in the optional argument to the document style.
\begin{center}
\begin{quote}
\verb"\documentstyle[times,jgrga]{article}"
\end{quote}
\end{center}
Please be aware that the ``times'' option uses a great deal of memory, and 
that the dvips program is rather difficult to install.  

\subsection{Miscellaneous}

When discussing atomic species, ionization levels 
can be indicated with the following command:
\begin{quote}
\verb"\ion{ELEMENT}{LEVEL}"
\end{quote}
The ionization state is specified as the second argument,
and should be given as a numeral.  For example, \ion{Ca}{3} 
is specified by typing \verb"\ion{Ca}{3}".

\begin{quote}
\verb"\case{NUM}{DENOM}"
\end{quote}

AGU\TeX\ contains a command that allows authors to specify an 
alternate form for fractions.  Authors will generally find it 
unnecessary to use any markup other than the standard \LaTeX\ 
\verb"\frac" command.  \LaTeX\ will set fractions in displayed 
math as built-up fractions.  However, it is sometimes desirable 
to use case fractions in displayed equations.  In such instances, 
one should use the \verb"\case" command rather than the \verb"\frac"
command.  A shilled fraction is produced without any special markup.
\begin{displaymath}
\renewcommand{\arraystretch}{1.4}
\begin{array}{llc}
\mbox{Built-up} & \verb"\frac{1}{2}" & \displaystyle\frac{1}{2} \\[.5ex]
\mbox{Case} & \verb"\case{1}{2}" & \case{1}{2} \\
\mbox{Shilled} & \verb"1/2" & 1/2 \\
\end{array}
\end{displaymath}

\subsection{Concluding the File}

The last thing in the electronic manuscript file should be the
\verb"\end{document}" command.  This command directs the formatter 
to perform assorted termination activities and finish processing 
the manuscript.

\section{Additional Documentation}

The preceding detailed explanation of the markup commands in this
package has certain merit, but many authors will prefer to examine
the sample papers that are included with the style files.  The 
files of interest are described below.

The file \verb"sample.tex" is a comprehensive example requiring nearly 
all of the capabilities of the package (in terms of markup as well as 
formatting) is in \verb"sample.tex".  This file is annotated with 
comments that describe the purpose of most of the markup.  The file 
\verb"sample.tex" includes examples of {\tt planotable} environments.

This user guide ({\tt\jobname.tex}) is also marked up with the 
AGU\TeX\ package, although it is not exemplary as a scientific paper.

A number of the markup commands described in the preceding
sections are standard \LaTeX\ commands, and the reader who is
unfamiliar with their syntax is referred to the \LaTeX\
manual [{\it Lamport}, 1985] for details.  A cribsheet listing 
all the \LaTeX\ commands (and some pertinent plain \TeX\ commands)
with short descriptions of each is published by the \TeX\ 
Users Group [{\it Botway and Biemesderfer}, 1989].

Authors who wish to know the ins and outs of \TeX\ itself should read 
the {\it\TeX book} [{\it Knuth}, 1984], probably more than once.  
There is a good deal of general information about typography in this 
source.  Many details of mathematical typography are discussed in a 
book by {\it Swanson} [1971].

%\newpage
\appendix
\section{Appendix A: Special Symbols}

The AGU\TeX\ package also contains a collection of assorted
macros for special symbols (or abbreviations) that authors 
tend to work out for themselves anyway.  Some of the 
definitions come from the {\it Astronomy and Astrophysics\/} 
package [{\it Springer}, 1989]; some are contributions from 
individuals.  We have tried to select a tractable number 
that were useful and also somewhat difficult to get right:
\begin{center}
\begin{tabular}{ll@{\hspace*{3em}}ll}
\verb"\deg" & \deg & 
\verb"\sq" & \sq \\
\verb"\sun" & \sun &
\verb"\earth" & \earth \\
\verb"\arcmin" & \arcmin & 
\verb"\arcsec" & \arcsec \\
\verb"\fd" & \fd & 
\verb"\fh" & \fh \\
\verb"\fm" & \fm & 
\verb"\fs" & \fs \\
\verb"\fdg" & \fdg & 
\verb"\farcm" & \farcm \\
\verb"\farcs" & \farcs & 
\verb"\fp" & \fp \\
\verb"\micron" & \micron & \verb"$\lesssim$" & $\lesssim$ \\
\end{tabular}
\end{center}
Most of these commands can be used in running text as well as 
when setting mathematical expressions.  The \verb"\lesssim" 
and \verb"\gtrsim" commands can only be used in math mode, 
which is sensible since they are relations.  It is possible 
to use \verb"\earth" and \verb"\sun" as subscripts, for 
example, \verb"$1.4 M_{\sun}$" yields $1.4 M_{\sun}$.

The \LaTeX\ language has a wide variety of special symbols for which 
markup commands have already been defined (see Tables A1-A12).  These 
range from diacritics to exotic mathematical operators.

This section groups \LaTeX's symbols together more or less according 
to function.  Some of these symbols are primarily for use in text;
most of them are mathematical symbols and can only be used in \LaTeX's 
math mode.  These tables are excerpted from the {\it \LaTeX\ Command 
Summary} [{\it Botway}, 1989].

\clearpage

\begin{center}
\setcounter{table}{0}\begin{planotable}{cccccc}
\tablewidth{30pc}
\tablecaption{Text-Mode Accents}
\tablehead{\multicolumn{1}{c}{Symbol} & 
\multicolumn{1}{c}{Command} & 
\multicolumn{1}{c}{Symbol} & 
\multicolumn{1}{c}{Command} & 
\multicolumn{1}{c}{Symbol} & 
\multicolumn{1}{c}{Command}}
\startdata \`{o} & \verb"\`{o}" & \={o} & \verb"\={o}" & \t{oo} & \verb" \t{oo}" \nl
\'{o} & \verb"\'{o}" & \.{o} & \verb"\.{o}" & \c{o}  & \verb"\c{o}" \nl
\^{o} & \verb"\^{o}" & \u{o} & \verb"\u{o}" & \d{o}  & \verb"\d{o}" \nl
\"{o} & \verb#\"{o}# & \v{o} & \verb"\v{o}" & \b{o}  & \verb"\b{o}" \nl
\~{o} & \verb"\~{o}" & \H{o} & \verb"\H{o}" & &
\end{planotable}

\begin{planotable}{ccclcl}
\tablewidth{30pc}
\tablecaption{National Symbols}
\tablehead{\multicolumn{1}{c}{Symbol} & 
\multicolumn{1}{c}{Command} & 
\multicolumn{1}{c}{Symbol} & 
\multicolumn{1}{l}{Command} & 
\multicolumn{1}{c}{Symbol} & 
\multicolumn{1}{l}{Command}}
\startdata \oe & \verb"\oe" & \aa & \verb"   \aa" & \l  & \verb"   \l" \nl
\OE & \verb"\OE" & \AA & \verb"   \AA" & \L  & \verb"   \L" \nl
\ae & \verb"\ae" & \o  & \verb"   \o"  & \ss & \verb"   \ss" \nl
\AE & \verb"\AE" & \O  & \verb"   \O"  & &
\end{planotable}

\begin{planotable}{clcccl}
\tablewidth{30pc}
\tablecaption{Miscellaneous Symbols}
\tablehead{
\multicolumn{1}{c}{Symbol} & 
\multicolumn{1}{l}{Command} & 
\multicolumn{1}{c}{Symbol} & 
\multicolumn{1}{c}{Command} & 
\multicolumn{1}{c}{Symbol} & 
\multicolumn{1}{l}{Command}}
\startdata
\dag  & \verb"  \dag"  & \S & \verb"\S" & \copyright & \verb"\copyright" \nl
\ddag & \verb"  \ddag" & \P & \verb"\P" & \pounds    & \verb"\pounds" \nl
\#    & \verb"  \#"    & \$ & \verb"\$" & \     & \verb"\%" \nl
\&    & \verb"  \&"    & \_ & \verb"\_" & & \nl
\{    & \verb"  \{"    & \} & \verb"\}" & &
\end{planotable}

\begin{planotable}{cl@{\hspace{4em}}cl}
\tablewidth{30pc}
\tablecaption{Math-Mode Accents}
\tablehead{
\multicolumn{1}{c}{Symbol} & 
\multicolumn{1}{l}{Command} & 
\multicolumn{1}{c}{Symbol} & 
\multicolumn{1}{l}{Command}}
\startdata
$\hat{a}$   & \verb"\hat{a}"   & $\dot{a}$   & \verb"\dot{a}"   \nl
$\check{a}$ & \verb"\check{a}" & $\ddot{a}$  & \verb"\ddot{a}"  \nl
$\tilde{a}$ & \verb"\tilde{a}" & $\breve{a}$ & \verb"\breve{a}" \nl
$\acute{a}$ & \verb"\acute{a}" & $\bar{a}$   & \verb"\bar{a}"   \nl
$\grave{a}$ & \verb"\grave{a}" & $\vec{a}$   & \verb"\vec{a}"
\end{planotable}

\clearpage

\begin{planotable}{cl@{\hspace{3em}}cl}
\tablewidth{30pc}
\tablecaption{Greek Letters (Math Mode)}
\tablehead{
\multicolumn{1}{c}{Symbol} & 
\multicolumn{1}{l}{Command} & 
\multicolumn{1}{c}{Symbol} & 
\multicolumn{1}{l}{Command}}
\startdata
$\alpha$   & \verb"\alpha"   & $\nu$      & \verb"\nu"      \nl
$\beta$    & \verb"\beta"    & $\xi$      & \verb"\xi"      \nl
$\gamma$   & \verb"\gamma"   & $o$        & \verb"o"        \nl
$\delta$   & \verb"\delta"   & $\pi$      & \verb"\pi"      \nl
$\epsilon$ & \verb"\epsilon" & $\rho$     & \verb"\rho"     \nl
$\zeta$    & \verb"\zeta"    & $\sigma$   & \verb"\sigma"   \nl
$\eta$     & \verb"\eta"     & $\tau$     & \verb"\tau"     \nl
$\theta$   & \verb"\theta"   & $\upsilon$ & \verb"\upsilon" \nl
$\iota$    & \verb"\iota"    & $\phi$     & \verb"\phi"     \nl
$\kappa$   & \verb"\kappa"   & $\chi$     & \verb"\chi"     \nl
$\lambda$  & \verb"\lambda"  & $\psi$     & \verb"\psi"     \nl
$\mu$      & \verb"\mu"      & $\omega$   & \verb"\omega"   \vspace{1em}\nl
$\varepsilon$ & \verb"\varepsilon" & $\varsigma$ & \verb"\varsigma" \nl
$\vartheta$   & \verb"\vartheta"   & $\varphi$   & \verb"\varphi"   \nl
$\varrho$     & \verb"\varrho"     & & \vspace{1em}\nl
$\Gamma$  & \verb"\Gamma"  & $\Sigma$   & \verb"\Sigma"   \nl
$\Delta$  & \verb"\Delta"  & $\Upsilon$ & \verb"\Upsilon" \nl
$\Theta$  & \verb"\Theta"  & $\Phi$     & \verb"\Phi"     \nl
$\Lambda$ & \verb"\Lambda" & $\Psi$     & \verb"\Psi"     \nl
$\Xi$     & \verb"\Xi"     & $\Omega$   & \verb"\Omega"   \nl
$\Pi$     & \verb"\Pi"     & &
\end{planotable}

\clearpage

\begin{planotable}{cl@{\hspace{3em}}cl}
\tablewidth{30pc}
\tablecaption{Binary Operations (Math Mode)}
\tablehead{
\multicolumn{1}{c}{Symbol} & 
\multicolumn{1}{l}{Command} & 
\multicolumn{1}{c}{Symbol} & 
\multicolumn{1}{l}{Command}}
\startdata
$\pm$       & \verb"\pm"       & $\cap$             & \verb"\cap"           \nl
$\mp$       & \verb"\mp"       & $\cup$             & \verb"\cup"           \nl
$\setminus$ & \verb"\setminus" & $\uplus$           & \verb"\uplus"         \nl
$\cdot$     & \verb"\cdot"     & $\sqcap$           & \verb"\sqcap"         \nl
$\times$    & \verb"\times"    & $\sqcup$           & \verb"\sqcup"         \nl
$\ast$      & \verb"\ast"      & $\triangleleft$    & \verb"\triangleleft"  \nl
$\star$     & \verb"\star"     & $\triangleright$   & \verb"\triangleright" \nl
$\diamond$  & \verb"\diamond"  & $\wr$              & \verb"\wr"            \nl
$\circ$     & \verb"\circ"     & $\bigcirc$         & \verb"\bigcirc"       \nl
$\bullet$   & \verb"\bullet"   & $\bigtriangleup$   & \verb"\bigtriangleup" \nl
$\div$      & \verb"\div"      & $\bigtriangledown$ & \verb"\bigtriangledown" \nl
$\lhd$      & \verb"\lhd"      & $\rhd$             & \verb"\rhd"           \nl
$\vee$      & \verb"\vee"      & $\odot$            & \verb"\odot"          \nl
$\wedge$    & \verb"\wedge"    & $\dagger$          & \verb"\dagger"        \nl
$\oplus$    & \verb"\oplus"    & $\ddagger$         & \verb"\ddagger"       \nl
$\ominus$   & \verb"\ominus"   & $\amalg$           & \verb"\amalg"         \nl
$\otimes$   & \verb"\otimes"   & $\unlhd$           & \verb"\unlhd"         \nl
$\oslash$   & \verb"\oslash"   & $\unrhd$           & \verb"\unrhd"
\end{planotable}

\begin{planotable}{cl@{\hspace{4em}}cl}
\tablewidth{30pc}
\tablecaption{Relations (Math Mode)}
\tablehead{
\multicolumn{1}{c}{Symbol} & 
\multicolumn{1}{l}{Command} & 
\multicolumn{1}{c}{Symbol} & 
\multicolumn{1}{l}{Command}}
\startdata
$\leq$        & \verb"\leq"        & $\geq$        & \verb"\geq"        \nl
$\prec$       & \verb"\prec"       & $\succ$       & \verb"\succ"       \nl
$\preceq$     & \verb"\preceq"     & $\succeq$     & \verb"\succeq"     \nl
$\ll$         & \verb"\ll"         & $\gg$         & \verb"\gg"         \nl
$\subset$     & \verb"\subset"     & $\supset$     & \verb"\supset"     \nl
$\subseteq$   & \verb"\subseteq"   & $\supseteq$   & \verb"\supseteq"   \nl
$\sqsubset$   & \verb"\sqsubset"   & $\sqsupset$   & \verb"\sqsupset"   \nl
$\sqsubseteq$ & \verb"\sqsubseteq" & $\sqsupseteq$ & \verb"\sqsupseteq" \nl
$\in$         & \verb"\in"         & $\ni$         & \verb"\ni"         \nl
$\vdash$      & \verb"\vdash"      & $\dashv$      & \verb"\dashv"      \nl
$\smile$      & \verb"\smile"      & $\mid$        & \verb"\mid"        \nl
$\frown$      & \verb"\frown"      & $\parallel$   & \verb"\parallel"   \nl
$\neq$        & \verb"\neq"        & $\perp$       & \verb"\perp"       \nl
$\equiv$      & \verb"\equiv"      & $\cong$       & \verb"\cong"       \nl
$\sim$        & \verb"\sim"        & $\bowtie$     & \verb"\bowtie"     \nl
$\simeq$      & \verb"\simeq"      & $\propto$     & \verb"\propto"     \nl
$\asymp$      & \verb"\asymp"      & $\models$     & \verb"\models"     \nl
$\approx$     & \verb"\approx"     & $\doteq$      & \verb"\doteq"      \nl
              &                    & $\Join$       & \verb"\Join"
\end{planotable}

\clearpage

\begin{planotable}{ccl@{\hspace{2em}}ccl}
\tablewidth{41pc}
\tablecaption{Variable-Sized Symbols (Math Mode)}
\tablehead{
\multicolumn{1}{c}{Symbol} & 
\multicolumn{1}{c}{Displaystyle} & 
\multicolumn{1}{l}{Command} & 
\multicolumn{1}{c}{Symbol} & 
\multicolumn{1}{c}{Displaystyle} &
\multicolumn{1}{l}{Command}} 
\startdata  
$\sum$      & $\displaystyle \sum$      & $\hbox{\verb"\sum"}$      & $\bigcap$    & $\displaystyle \bigcap$    & $\hbox{\verb"\bigcap"}$ \vspace{4pt}\nl 
$\prod$     & $\displaystyle \prod$     & $\hbox{\verb"\prod"}$     & $\bigcup$    & $\displaystyle \bigcup$    & $\hbox{\verb"\bigcup"}$ \vspace{4pt}\nl 
$\coprod$   & $\displaystyle \coprod$   & $\hbox{\verb"\coprod"}$   & $\bigsqcup$  & $\displaystyle \bigsqcup$  & $\hbox{\verb"\bigsqcup"}$ \vspace{4pt}\nl 
$\int$      & $\displaystyle \int$      & $\hbox{\verb"\int"}$      & $\bigvee$    & $\displaystyle \bigvee$    & $\hbox{\verb"\bigvee"}$ \vspace{4pt}\nl 
$\oint$     & $\displaystyle \oint$     & $\hbox{\verb"\oint"}$     & $\bigwedge$  & $\displaystyle \bigwedge$  & $\hbox{\verb"\bigwedge"}$ \vspace{4pt}\nl 
$\bigodot$  & $\displaystyle \bigodot$  & $\hbox{\verb"\bigodot"}$  & $\bigotimes$ & $\displaystyle \bigotimes$ & $\hbox{\verb"\bigotimes"}$ \vspace{4pt}\nl 
$\bigoplus$ & $\displaystyle \bigoplus$ & $\hbox{\verb"\bigoplus"}$ & $\biguplus$  & $\displaystyle \biguplus$  & $\hbox{\verb"\biguplus"}$
\end{planotable}

\begin{planotable}{cl@{\hspace{2em}}cl}
\tablewidth{30pc}
\tablecaption{Delimiters (Math Mode)}
\tablehead{
\multicolumn{1}{l}{Symbol} & 
\multicolumn{1}{l}{Command} & 
\multicolumn{1}{l}{Symbol} & 
\multicolumn{1}{l}{Command}}
\startdata
$($            & \verb"("            & $)$            & \verb")" \nl
$[$            & \verb"["            & $]$            & \verb"]" \nl
$\{$           & \verb"\{"           & $\}$           & \verb"\}" \nl
$\lfloor$      & \verb"\lfloor"      & $\rfloor$      & \verb"\rfloor" \nl
$\lceil$       & \verb"\lceil"       & $\rceil$       & \verb"\rceil" \nl
$\langle$      & \verb"\langle"      & $\rangle$      & \verb"\rangle" \nl
$/$            & \verb"/"            & $\backslash$   & \verb"\backslash" \nl
$\vert$        & \verb"\vert"        & $\Vert$        & \verb"\Vert" \nl
$\uparrow$     & \verb"\uparrow"     & $\Uparrow$     & \verb"\Uparrow" \nl
$\downarrow$   & \verb"\downarrow"   & $\Downarrow$   & \verb"\Downarrow" \nl
$\updownarrow$ & \verb"\updownarrow" & $\Updownarrow$ & \verb"\Updownarrow"
\end{planotable}

\begin{planotable}{llll}
\tablewidth{30pc}
\tablecaption{Function Names (Math Mode)}
\startdata
\verb"\arccos"  &   \verb"\csc"  &  \verb"\ker"     &  \verb"\min"\nl
\verb"\arcsin"  &   \verb"\deg"  &  \verb"\lg"      &  \verb"\Pr"\nl
\verb"\arctan"  &   \verb"\det"  &  \verb"\lim"     &  \verb"\sec"\nl
\verb"\arg"     &   \verb"\dim"  &  \verb"\liminf"  &  \verb"\sin"\nl
\verb"\cos"     &   \verb"\exp"  &  \verb"\limsup"  &  \verb"\sinh"\nl
\verb"\cosh"    &   \verb"\gcd"  &  \verb"\ln"      &  \verb"\sup"\nl
\verb"\cot"     &   \verb"\hom"  &  \verb"\log"     &  \verb"\tan"\nl
\verb"\coth"    &   \verb"\inf"  &  \verb"\max"     &  \verb"\tanh"
\end{planotable}

\clearpage

\begin{planotable}{clcl}
\tablewidth{30pc}
\tablecaption{Arrows (Math Mode)}
\tablehead{
\multicolumn{1}{c}{Symbol} &
\multicolumn{1}{l}{Command} & 
\multicolumn{1}{c}{Symbol} & 
\multicolumn{1}{l}{Command}}
\startdata
$\leftarrow$         & \verb"\leftarrow"         & $\longleftarrow$        & \verb"\longleftarrow" \nl
$\Leftarrow$         & \verb"\Leftarrow"         & $\Longleftarrow$        & \verb"\Longleftarrow" \nl
$\rightarrow$        & \verb"\rightarrow"        & $\longrightarrow$       & \verb"\longrightarrow" \nl
$\Rightarrow$        & \verb"\Rightarrow"        & $\Longrightarrow$       & \verb"\Longrightarrow" \nl
$\leftrightarrow$    & \verb"\leftrightarrow"    & $\longleftrightarrow$   & \verb"\longleftrightarrow" \nl
$\Leftrightarrow$    & \verb"\Leftrightarrow"    & $\Longleftrightarrow$   & \verb"\Longleftrightarrow" \nl
$\mapsto$            & \verb"\mapsto"            & $\longmapsto$           & \verb"\longmapsto" \nl
$\hookleftarrow$     & \verb"\hookleftarrow"     & $\hookrightarrow$       & \verb"\hookrightarrow" \nl
$\leftharpoonup$     & \verb"\leftharpoonup"     & $\rightharpoonup$       & \verb"\rightharpoonup" \nl
$\leftharpoondown$   & \verb"\leftharpoondown"   & $\rightharpoondown$     & \verb"\rightharpoondown" \nl
$\rightleftharpoons$ & \verb"\rightleftharpoons" & $\leadsto$              & \verb"\leadsto" \nl
$\uparrow$           & \verb"\uparrow"           & $\Updownarrow$          & \verb"\Updownarrow"\nl
$\Uparrow$           & \verb"\Uparrow"           & $\nearrow$              & \verb"\nearrow"\nl
$\downarrow$         & \verb"\downarrow"         & $\searrow$              & \verb"\searrow" \nl
$\Downarrow$         & \verb"\Downarrow"         & $\swarrow$              & \verb"\swarrow"\nl
$\updownarrow$       & \verb"\updownarrow"       & $\nwarrow$              & \verb"\nwarrow"
\end{planotable}

\begin{planotable}{cl@{\hspace{3em}}cl}
\tablecaption{Miscellaneous Symbols (Math Mode)}
\tablehead{
\multicolumn{1}{c}{Symbol} & 
\multicolumn{1}{l}{Command} & 
\multicolumn{1}{c}{Symbol} & 
\multicolumn{1}{l}{Command}}
\startdata
$\aleph$   & \verb"\aleph"   & $\prime$       & \verb"\prime"       \nl
$\hbar$    & \verb"\hbar"    & $\emptyset$    & \verb"\emptyset"    \nl
$\imath$   & \verb"\imath"   & $\nabla$       & \verb"\nabla"       \nl
$\jmath$   & \verb"\jmath"   & $\surd$        & \verb"\surd"        \nl
$\ell$     & \verb"\ell"     & $\top$         & \verb"\top"         \nl
$\wp$      & \verb"\wp"      & $\bot$         & \verb"\bot"         \nl
$\Re$      & \verb"\Re"      & $\|$           & \verb"\|"           \nl
$\Im$      & \verb"\Im"      & $\angle$       & \verb"\angle"       \nl
$\partial$ & \verb"\partial" & $\triangle$    & \verb"\triangle"    \nl
$\infty$   & \verb"\infty"   & $\backslash$   & \verb"\backslash"   \nl
$\Box$     & \verb"\Box"     & $\Diamond$     & \verb"\Diamond"     \nl
$\forall$  & \verb"\forall"  & $\sharp$       & \verb"\sharp"       \nl
$\exists$  & \verb"\exists"  & $\clubsuit$    & \verb"\clubsuit"    \nl
$\neg$     & \verb"\neg"     & $\diamondsuit$ & \verb"\diamondsuit" \nl
$\flat$    & \verb"\flat"    & $\heartsuit$   & \verb"\heartsuit"   \nl
$\natural$ & \verb"\natural" & $\spadesuit$   & \verb"\spadesuit"   \nl
$\mho$     & \verb"\mho"     & & 
\end{planotable}
\end{center}

\clearpage

\begin{references}
\reference Abt, H., \apj, {\it 357}, 1, 1990.

\reference Botway, L., and C. Biemesderfer, {\it \LaTeX\ Command Summary},
\TeX\ Users Group, Providence, R. I., 1989.
 
\reference Knuth, D., {\it The \TeX book}, Addison-Wesley, Reading, 
	Mass., 1984.

\reference Lamport, L., {\it \LaTeX: A Document Preparation System\/},
	Addison-Wesley, Reading, Mass., 1985.

\reference Springer-Verlag, {\it Springer-Verlag \TeX\ AA Macro \protect\\ 
        Package}, Springer-Verlag, Heidelberg, Germany, 1989.

\reference Springer-Verlag, {\it Springer-Verlag \LaTeX\ AA Macro Package}, 
	Springer-Verlag, Heidelberg, Germany, 1990.

\reference Swanson, E., \, {\it Mathematics \, Into \, Type}, \, American 
	Math\-ematical Society, Providence, R. I., 1979.
\end{references}

\end{document}
