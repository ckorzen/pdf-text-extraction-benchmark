%%%%%%%%%% espcrc1.tex %%%%%%%%%%
\documentstyle[12pt,twoside,fleqn,espcrc1]{article}

% put your own definitions here:
%   \newcommand{\cZ}{\cal{Z}}
%   \newtheorem{def}{Definition}[section]
%   ...
\newcommand{\ttbs}{\char'134}
\newcommand{\AmS}{{\protect\the\textfont2
  A\kern-.1667em\lower.5ex\hbox{M}\kern-.125emS}}

% add words to TeX's hyphenation exception list
\hyphenation{author another created financial paper re-commend-ed}

% declarations for front matter
\title{Elsevier instructions for the preparation of a 
       camera-ready paper in \LaTeX}

\author{P. de Groot\address{Mathematics and Computer Science Division, 
        Elsevier Science Publishers B.V., \\ 
        P.O. Box 103, 1000 AC Amsterdam, The Netherlands}%
        \thanks{Footnotes should appear on the first page only to
                indicate your present address (if different from your
                normal address), research grant, sponsoring agency, etc.
                These are obtained with the {\tt\ttbs thanks} command.}
        and 
        X.-Y. Wang\address{Economics Department, University of Winchester, \\
        2 Finch Road, Winchester, Hampshire P3L T19, United Kingdom}}



\begin{document}
% typeset front matter
\maketitle

\begin{abstract}
These pages provide you with an example of the layout and style which
we wish you to adopt during the preparation of your paper. Your text
will be photographically reduced by 20--25\%. This is the output from
the \LaTeX{} document style you requested.
\end{abstract}

\section{FORMAT}

Text should be produced within the dimensions shown on these pages:
total width of 16~cm and a maximum length of 21~cm on first pages and
23~cm on second and following pages. The \LaTeX{} document style uses
the maximal stipulated length apart from the following two exceptions
(i) \LaTeX{} does not begin a new section directly at the bottom of a
page, but transfers the heading to the top of the next page; (ii)
\LaTeX{} never (well, hardly ever) exceeds the length of the text area
in order to complete a section of text or a paragraph.
Here are some references: \cite{Scho70,Mazu84}.

\subsection{Spacing}

We normally recommend the use of 1.0 (single) line spacing. However,
when typing complicated mathematical text \LaTeX{} automatically
increases the space between text lines in order to prevent sub- and
superscript fonts overlapping one another and making your printed
matter illegible.

\subsection{Fonts}

These instructions have been produced using a 12 point Computer Modern
Roman. Other recommended fonts are 12 point Times Roman, New Century
Schoolbook, Bookman Light and Palatino.

\section{PRINTOUT}

The most suitable printer is a laser printer. A dot matrix printer
should only be used if it possesses an 18 or 24 pin printhead
(``letter-quality'').

The printout submitted should be an original; a photocopy is not
acceptable. Please make use of good quality plain white A4 (or US
Letter) paper size. {\em The dimensions shown here should be strictly
adhered to: do not make changes to these dimensions, which are
determined by the document style}. The document style leaves at least
3~cm at the top of the page before the head, which contains the page
number.

Printers sometimes produce text which contains light and dark streaks,
or has considerable lighting variation either between left-hand and
right-hand margins or between text heads and bottoms. To achieve
optimal reproduction quality, the contrast of text lettering must be
uniform, sharp and dark over the whole page and throughout the article.

If corrections are made to the text, print completely new replacement
pages. The contrast on these pages should be consistent with the rest
of the paper as should text dimensions and font sizes.

\section{TABLES AND ILLUSTRATIONS}

Tables should be made with \LaTeX; illustrations should be originals or
sharp prints. They should be arranged throughout the text and
preferably be included {\em on the same page as they are first
discussed}. They should have a self-contained caption and be positioned
in flush-left alignment with the text margin. Two small illustrations
may be placed alongside one another as shown with
Figures~\ref{fig:largenenough} and~\ref{fig:toosmall}. All
illustrations will undergo the same reduction as the text.

\subsection{Tables}

Tables should be presented in the form shown in
Table~\ref{tab:effluents}.  Their layout should be consistent
throughout.

\begin{table}[hbt]
% -----------------------------------------------------
% adapted from TeX book, p. 241
\newlength{\digitwidth} \settowidth{\digitwidth}{\rm 0}
\catcode`?=\active \def?{\kern\digitwidth}
% -----------------------------------------------------
\caption{Biologically treated effluents (mg/l)}
\label{tab:effluents}
\begin{tabular*}{\textwidth}{@{}l@{\extracolsep{\fill}}rrrr}
\hline
                 & \multicolumn{2}{l}{Pilot plant} 
                 & \multicolumn{2}{l}{Full scale plant} \\
\cline{2-3} \cline{4-5}
                 & \multicolumn{1}{r}{Influent} 
                 & \multicolumn{1}{r}{Effluent} 
                 & \multicolumn{1}{r}{Influent} 
                 & \multicolumn{1}{r}{Effluent}         \\
\hline
Total cyanide    & $ 6.5$ & $0.35$ & $  2.0$ & $  0.30$ \\
Method-C cyanide & $ 4.1$ & $0.05$ &         & $  0.02$ \\
Thiocyanide      & $60.0$ & $1.0?$ & $ 50.0$ & $ <0.10$ \\
Ammonia          & $ 6.0$ & $0.50$ &         & $  0.10$ \\
Copper           & $ 1.0$ & $0.04$ & $  1.0$ & $  0.05$ \\
Suspended solids &        &        &         & $<10.0?$ \\
\hline
\multicolumn{5}{@{}p{120mm}}{Reprinted from: G.M. Ritcey,
                             Tailings Management,
                             Elsevier, Amsterdam, 1989, p. 635.}
\end{tabular*}
\end{table}

Horizontal lines should be placed above and below table headings, above
the subheadings and at the end of the table above any notes. Vertical
lines should be avoided.

If a table is too long to fit onto one page, the table number and
headings should be repeated above the continuation of the table. For
this you have to reset the table counter with
\verb|\addtocounter{table}{-1}|. Alternatively, the table can be turned
by $90^\circ$ (`landscape mode') and spread over two consecutive pages
(first an even-numbered, then an odd-numbered one) created by means of
\verb|\begin{table}[h]| without a caption. To do this, you prepare the
table as a separate \LaTeX{} document and attach the tables to the
empty pages with a few spots of suitable glue.

\subsection{Line drawings}

Line drawings should be drawn in India ink on tracing paper with the
aid of a stencil or should be glossy prints of the same; computer
prepared drawings are also acceptable. They should be attached to your
manuscript page, correctly aligned, using suitable glue and {\em not
transparent tape}. When placing a figure at the top of a page, the top
of the figure should be at the same level as the bottom of the first
text line.

All notations and lettering should be no less than 2.5\,mm high. The use
of heavy black, bold lettering should be avoided as this will look
unpleasantly dark when printed.

\begin{figure}[htb]
\begin{minipage}[t]{80mm}
\framebox[79mm]{\rule[-26mm]{0mm}{52mm}}
\caption{Good sharp prints should be used and not (distorted) photocopies.}
\label{fig:largenenough}
\end{minipage}
%
\hspace{\fill}
%
\begin{minipage}[t]{75mm}
\framebox[74mm]{\rule[-26mm]{0mm}{52mm}}
\caption{Remember to keep details clear and large enough to
         withstand a 20--25\% reduction.}
\label{fig:toosmall}
\end{minipage}
\end{figure}

\subsection{Black and white photographs}

Photographs must always be sharp originals ({\em not screened
versions\/}) and rich in contrast. They will undergo the same reduction
as the text and should be pasted on your page in the same way as line
drawings.

\subsection{Colour photographs}

Sharp originals ({\em not transparencies or slides\/}) should be
submitted close to the size expected in publication. Charges for the
processing and printing of colour will be passed on to the author(s) of
the paper. As costs involved are per page, care should be taken in the
selection of size and shape so that two or more illustrations may be
fitted together on one page. Please contact the Technical Editor in the
Camera-Ready Publications Department at Elsevier for a price quotation
and layout instructions before producing your paper in its final form.

\section{EQUATIONS}

Equations should be flush-left with the text margin; \LaTeX{} ensures
that the equation is preceded and followed by one line of white space. 
\LaTeX{} provides the document-style option {\tt fleqn} to get the
flush-left effect.

\begin{equation}
H_{\alpha\beta}(\omega) = E_\alpha^{(0)}(\omega) \delta_{\alpha\beta} +
                          \langle \alpha | W_\pi | \beta \rangle 
\end{equation}

You need not put in equation numbers, since this is taken care of
automatically. The equation numbers are always consecutive and are
printed in parentheses flush with the right-hand margin of the text and
level with the last line of the equation. For multi-line equations, use
the {\tt eqnarray} environment. For complex mathematics, use the
\AmS-\LaTeX{} package.

\begin{thebibliography}{9}
\bibitem{Scho70} S. Scholes, Discuss. Faraday Soc. No. 50 (1970) 222.
\bibitem{Mazu84} O.V. Mazurin and E.A. Porai-Koshits (eds.),
                 Phase Separation in Glass, North-Holland, Amsterdam, 1984.
\bibitem{Dimi75} Y. Dimitriev and E. Kashchieva, 
                 J. Mater. Sci. 10 (1975) 1419.
\bibitem{Eato75} D.L. Eaton, Porous Glass Support Material,
                 US Patent No. 3 904 422 (1975).
\end{thebibliography}

References should be collected at the end of your paper. Do not begin
them on a new page unless this is absolutely necessary. They should be
prepared according to the sequential numeric system making sure that
all material mentioned is generally available to the reader. Use
\verb+\cite+ to refer to the entries in the bibliography so that your
accumulated list corresponds to the citations made in the text body. 

Above we have listed some references according to the
sequential numeric system \cite{Scho70,Mazu84,Dimi75,Eato75}.
\end{document}



